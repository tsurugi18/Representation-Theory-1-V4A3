\documentclass{article}

\usepackage{amsmath}
\usepackage{amssymb}
\usepackage{amsthm}
\usepackage{enumerate}
\usepackage{bbm}
\usepackage{lipsum}
\usepackage{fancyhdr}
\usepackage{calrsfs}
\usepackage{tikz-cd} 

\newtheorem{theorem}{Theorem}[section] 
\newtheorem{proposition}{Proposition}[section] 
\newtheorem{definition}{Definition}[section] 
\newtheorem{lemma}{Lemma}[section] 
\newtheorem{notation}{Notation}[section] 
\newtheorem{remark}{Remark}[section] 
\newtheorem{corollary}{Corollary}[section] 
\newtheorem{terminology}{Terminology}[section] 
\newtheorem{example}{Example}[section] 
\numberwithin{equation}{section}

\DeclareMathOperator{\diam}{diam}
\DeclareMathOperator{\rk}{rk}
\DeclareMathOperator{\rank}{rank}
\DeclareMathOperator{\Isom}{Isom}
\DeclareMathOperator{\Hom}{Hom}
\DeclareMathOperator{\Dom}{Dom}
\DeclareMathOperator{\grad}{grad}
\DeclareMathOperator{\Span}{Span}
\DeclareMathOperator{\interior}{int}
\DeclareMathOperator{\ind}{ind}
\DeclareMathOperator{\supp}{supp}
\DeclareMathOperator{\sgn}{sgn}
\DeclareMathOperator{\ob}{ob}
\DeclareMathOperator{\Spec}{Spec}
\DeclareMathOperator{\PreSh}{PreSh}
\DeclareMathOperator{\Fun}{Fun}
\DeclareMathOperator{\Ker}{Ker}
\DeclareMathOperator{\Image}{Im}
\DeclareMathOperator{\Ad}{Ad}
\DeclareMathOperator{\ad}{ad}
\DeclareMathOperator{\End}{End}
\DeclareMathOperator{\GL}{GL}
\DeclareMathOperator{\SL}{SL}
\DeclareMathOperator{\SU}{SU}
\DeclareMathOperator{\Lie}{Lie}
\DeclareMathOperator{\tr}{tr}
\DeclareMathOperator{\Der}{Der}
\DeclareMathOperator{\Aut}{Aut}
\DeclareMathOperator{\rad}{rad}
\DeclareMathOperator{\id}{id}

\newcommand{\tens}[1]{%
  \mathbin{\mathop{\otimes}\displaylimits_{#1}}%
}

\title{Representation Theory 1 V4A3}
\author{So Murata}
\date{2024/2025 Winter Semester - Uni Bonn}

\begin{document}
\maketitle



\section{Overview of the material}

\subsection{Lie groups}

\begin{definition}
A Lie group is a group $G$ whose underlying set is endowed with the structure of smooth manifolds such that multiplication and inversions are smooth maps.
\end{definition}

\begin{definition}
A topological group is a group $G$ whose underlying set is endowed with the structure of topological space such that multiplication and inversions are continuous.
\end{definition}

\section{Preliminaries}

\subsection{Topology}

\begin{definition}
We have two axioms about the topological spaces
\begin{enumerate}
\item $T_0$(Komogolov) : Given any 2 points, there exists an open set such that it contains one of them but not both.
\item $T_1$(Hausdorff) : Given any 2 points, there exist disjoints open set that each contains one of them.
\end{enumerate}
\end{definition}

\begin{definition}
A topological space is second countable if it has a basis which contains at most countably many subsets.
\end{definition}

\begin{definition}
Let $X,Y$ be topological spaces. A continuous function $f:X\to Y$ is said to be proper if the preimage of arbitrary compact set in $Y$ is again compact.
\end{definition}

\begin{definition}
Let $X$ be a topological space. A connected component $C$ of $x\in X$ is the largest connected set which contains $x$.
\end{definition}

\begin{proposition}
If $C$ is a connected component of the topological space $X$, then it is closed.
\label{comp_clo}
\end{proposition}

\begin{proof}
Let $f:\overline{C}\to\{0,1\}$ be a continuous function where $\{0,1\}$ is  with the discrete topology. Then for any $x\in C$ we conclude $f(x)=0$ without the loss of generality. By the continuity of $f$ we conclude that $f(x)=0$ for any $x\in \overline{C}$. 
\end{proof}

\begin{definition}
A topological space $(X,\mathcal{T})$ is said to be locally connected if for any point $x\in X$ and its neighborhood $U$, there exists a connected neighborhood $V$ such that $x\in V\subset U$.
\end{definition}

\begin{proposition}
A component of locally connected topological space is open.
\label{comp_op_loc}
\end{proposition}

\begin{proof}
%later
\end{proof}

\subsection{Linear Algebra}

\begin{definition}
Let $V,W$ be finite dimensional $\mathbb{K}$-vector space. The duality pairing is a triplet $(V,W,\langle\cdot,\cdot\rangle)$ where 
\begin{equation*}
\langle\cdot,\cdot\rangle:V\times W\to\mathbb{K}
\end{equation*}
is a bilinear form such that
\begin{equation*}
\forall v\in W, \forall w\in W, \langle v,w\rangle = 0\Rightarrow v = 0,
\end{equation*}
and similarly for $\langle\cdot,w\rangle$.
\end{definition}

\begin{proposition}
Let $(V,W,\langle\cdot,\cdot\rangle)$ be a duality pairing. Then we have the following isomorphisms
\begin{equation*}
V\cong W^*,\quad W\cong V^*.
\end{equation*}
\end{proposition}

\begin{lemma}
Let $(V,W,\langle\cdot,\cdot\rangle)$ be a duality pairing over $\mathbb{R}$. Then we can extend this to a duality pairing over $\mathbb{C}$ by
\begin{equation*}
(V\tens{\mathbb{R}}\mathbb{C})\times (W\tens{\mathbb{R}}\mathbb{C})\to\mathbb{C},
\end{equation*}
\end{lemma}

\begin{corollary}
Let $(V,W,\langle\cdot,\cdot\rangle)$ be a duality pairing over $\mathbb{R}$ then we have
\begin{equation*}
(V\tens{\mathbb{R}}\mathbb{C})^*\cong W\tens{\mathbb{R}}\mathbb{C}\cong V^*\tens{\mathbb{R}}\mathbb{C}.
\end{equation*}
\end{corollary}

\begin{definition}
Let $V$ be a finite dimensional $\mathbb{R}$ vector space. A subset $R\subseteq V^*$ is called a root system of rank $\dim V$ consisting of roots if it generates $V^*$, for each $\alpha,\beta\in R$, there is a $\alpha^\lor$ such that
\begin{enumerate}
\item $\langle\alpha^\lor,\alpha\rangle = 2$,
\item $s_\alpha:V^*\to V^*$, $s_\alpha(x) = x-\langle\alpha^\lor,x\rangle x$, $s_\alpha(R)\subseteq R$,
\item $\langle\alpha^\lor,\beta\rangle\in\mathbb{Z}$.
\end{enumerate}
And such $\alpha^\lor$ is called a coroot of $\alpha$.
\end{definition}

\begin{definition}
A root system $R$ of a vector space $V$ is reduced if for any $\alpha\in R$, we have
\begin{equation*}
\forall c\in\mathbb{R}, c\alpha\in R\Rightarrow c = \pm1.
\end{equation*}
\end{definition}

\begin{lemma}
A root system $R$ of a vector space $V$. Each root has a unique coroot.
\end{lemma}

\begin{remark}
Denote hyperplanes
\begin{equation*}
H_\alpha = \Ker(\langle \cdot,\alpha\rangle)\subseteq V, \quad H_{\alpha^\lor} = \Ker(\langle\alpha^\lor,\cdot\rangle)\subseteq V^*.
\end{equation*}
By the definition, $\alpha^\lor\not\in H_\alpha,\alpha^\lor\not\in H_{\alpha^\lor}$.\\
\par Observe that for any $x\in H_{\alpha^\lor}$, we have
\begin{equation*}
s_\alpha(x) = x.
\end{equation*}
While for $\alpha$ we have
\begin{equation*}
s_\alpha(\alpha) = -\alpha.
\end{equation*}
From this observation, we conclude that $s_\alpha$ is a reflection onto the hyperplane $H_{\alpha^\lor}$.
\end{remark}

\begin{definition}
Let $R$ be a root system of a vector space $V$. For $\alpha\in V$ we denote,
\begin{equation*}
H_\alpha = \Ker(\langle \cdot,\alpha\rangle)\subseteq V, \quad H_{\alpha^\lor} = \Ker(\langle\alpha^\lor,\cdot\rangle)\subseteq V^*
\end{equation*}
which are called the root hyperplanes.
\end{definition}

\begin{definition}
Let $R$ be a root system of a vector space $V$. The connected components of 
\begin{equation*}
V\backslash\bigcup_{\alpha\in R}H_\alpha
\end{equation*}
are called the Weyl chambers.
\end{definition}

\begin{definition}
Let $V_1,V_2$ be finite dimensional vector spaces and $R_1,R_2$ be their root systems, respectively.\\
\par An isomorphism of root systems $f:V_1\to V_2$ is an isomorphism of vector spaces which satisfies that 
\begin{equation*}
f(R_1) = R_2, \quad \forall \alpha,\beta\in R_1, \langle f(\alpha)^\lor,f(\beta)\rangle = \langle\alpha^\lor,\beta\rangle.%check
\end{equation*}
\end{definition}

\begin{proposition}
Let $V$ be a finite dimensional vector space and $R$ be its root system. Then 
\begin{equation*}
R^\lor = \{\alpha^\lor\:|\: \alpha\in R\}
\end{equation*}
is a root system in $V$.
\end{proposition}

\begin{remark}
Such $R^\lor$ is called a dual root system of $V$.
\end{remark}

\begin{definition}
The Weyl group is a finite subgroup of $\Aut(V^*)$ generated by the reflections $\{s_\alpha\in\Aut(V^*)\:|\: \alpha\in R\}$. 
\end{definition}

\begin{proposition}
Let $W$ be a Weyl group then under the isomorpshim $\varphi:\Aut(V)\to\Aut(V^*)$,
\begin{equation*}
\varphi(f) = f^{*,-1}
\end{equation*}
The Weyl group of $R$ and $R^\lor$ are identified. 
\end{proposition}

\begin{proposition}
Let $(V,\langle\cdot,\cdot\rangle)$ be a scalar product space which is invariant under the Weyl group $W$. Then it gives a $W$-equivalent isomorphism $V\to V^*$ such that 
\begin{equation*}
\alpha^\lor\mapsto {\frac {2\alpha} {\langle\alpha,\alpha\rangle}}.
\end{equation*}
\end{proposition}

\begin{theorem}
Choose a Weyl chambers $C$ and define
\begin{equation*}
P = \{\alpha\in R\:|\: \alpha(c)>0\}.
\end{equation*}
Then we have
\begin{equation*}
R=P\cup-P.
\end{equation*}
Furthermore, let us define $\Delta\subseteq P$ to be such that 
\begin{equation*}
\Delta = \{\alpha\in P\:|\: \forall \beta,\gamma\in P, \alpha\not=\beta+\gamma\}.
\end{equation*}
Then $\Delta$ is a basis of $V^*$.
\end{theorem}

\begin{definition}
Such $P$ is called a set of positive roots and $\Delta$ is called a set of simple roots.
\end{definition}

\begin{remark}
There exist bijections between these three sets
\begin{equation*}
\{\text{Weyl chambers}\}\leftrightarrow\{\text{Set of positive roots}\}\leftrightarrow\{\text{Sets of simple roots}\}
\end{equation*}
\end{remark}

\begin{example}
Let us take $V=V^*=\mathbb{R}$. Then we have
\begin{equation*}
R = \{a,-a\},a\in\mathbb{R}^\times, R^\lor = \{{\frac 2 a}, -{\frac 2 a}\}.
\end{equation*}
With Weyl group
\begin{equation*}
W = \mathbb{Z}/2\mathbb{Z}.
\end{equation*}
We have the reflection is 
\begin{equation*}
s_a(x) = -x.
\end{equation*}
\end{example}

\begin{example}
Let $V=\mathbb{R}^2_0 = \{(x,y)\in\mathbb{R}^2\:|\: x+y=0\}$.  Restricting the Euclidean scalar product on $\mathbb{R}^2$ to $V$, we get $V^*\cong V$. We now have
\end{example}

\subsection{Multi-linear forms}
\begin{definition}
Let $\mathbb{V}$ be a vector space and $\varphi:\bigoplus_{i=1}^mV\to\mathbb{R}$ is called a $m$-multi-linear function if for any $i=1,\cdots,m$ and $\{a_j\}_{j\not=i}\subset V$ we have
\begin{equation*}
\varphi(a_1,\cdots,a_{i-1},x,a_{i+1},\cdots,a_m):V\to\mathbb{R}
\end{equation*}
is a linear function
\end{definition}

\begin{definition}
Let $X$ be a smooth $n$-dimensional manifold and $m\in\mathbb{N}$. Then we define the followings
\begin{enumerate}
\item $\mathcal{L}^m_p=\{\varphi:\bigoplus_{i=1}^mT_pX\to\mathbb{R}|\varphi \text{ is a m-multi-linear function.}\}$
\item $\mathcal{L}^m=\bigcup_{p\in X}\mathcal{L}^m_p$
\end{enumerate}
\end{definition}

\begin{definition}
Let $X$ be a smooth $n$-dimensional manifold. A map $V:X\to\mathcal{L}^m$ is called a $m$-tensorfield if 
\begin{enumerate}[i.]
\item For any $p\in X$, $V(p)\in\mathcal{L}^m_p$.
\item For any chart $(U,\varphi)$ around $p$ with a basis $\{e_1^\varphi,\cdots,e_n^\varphi\}$ and for any $i_1,\cdots,i_m\in\{1,\cdots,n\}$ we have a map $V_{(i_1,\cdots,i_m)}:X\to\mathbb{R}$ such that $V_{(i_1,\cdots,i_m)}(p)=V(p)(\underline{e}_{i_1},\cdots,\underline{e}_{i_m})$ is smooth.
\end{enumerate}
\end{definition}

\begin{proposition}
For any m tensorfield $V$, we have
\end{proposition}

\begin{definition}
We define $\mathcal{V}^m(X)$ to be the set of all $m$-tensorfield.
\end{definition}

\begin{proposition}
$\mathcal{V}^m(X)$ is a vector space over $\mathbb{R}$ and a module over $\mathcal{F}(X)$ with the common basis $\{E_{i_1,\cdots,i_m}\}_{i_1,\cdots,i_m\in\{1,\cdots,n\}}$
\end{proposition}

\begin{proposition}
Let $X$ be a smooth $n$-dimensional manifold and $V:X\to\mathcal{L}^m$ be such that for any $p\in X, V(p)\in\mathcal{L}_p^m$ the followings are equivalent.
\begin{enumerate}
\item $V$ is a $m$-tensorfield.
\item For any chart $(U,\varphi)$ around $p$ with basis $\{\underline{e}_1^\varphi,\cdots,\underline{e}_n^\varphi\}$ and for any $1\leq i_1,\cdots,i_m\leq n$ there exist smooth mappings $\lambda_{i_1,\cdots,i_m}:X\to\mathbb{R}$ such that $V(p)=\sum_{1\leq i_1,\cdots,i_m\leq n} \lambda_{i_1,\cdots,i_m}(p)E_{i_1,\cdots,i_m}^\varphi$. 
\item For any vectorfields $v_1,\cdots,v_m:X\to TX$ we have a function $V:X\to\mathbb{R}$ such that $V_{v_1,\cdots,v_m}(p)=V(p)(v_1(p),\cdots,v_m(p))$ is smooth.
\end{enumerate}
\end{proposition}

\begin{proof}
1.$\Leftrightarrow$2. is trivial. 1.$\Rightarrow$3. is clear by the multi-linearity, and 3.$\Rightarrow$1. is choosing $v_i=e_i^\varphi$ for each $i=1,\cdots,n$.\\
\end{proof}
\begin{proposition}
Let $V:X\to\mathcal{L}^m$ then thne followings are equivalent. 
\begin{enumerate}
\item $V$ is a $m$-tensorfield.
\item For any $\{v_1,\cdots,v_m\}\in\mathcal{V}(X)$, $\Psi:\bigoplus_{i=1}^m\mathcal{V}(X)\to\mathcal{F}(X)$ such that $\Psi(v_1,\cdots,v_m)(p)=V(p)(v_1(p),\cdots,v_m(p))$ is smooth and $\mathcal{F}(X)$-linear.
\end{enumerate}
\end{proposition}

\begin{proof}
1.$\Rightarrow$2. follows from the multilinearity and decompositions of tensors. 2.$\Rightarrow$1. follows by fixing all element except one we still have the linearity thus, the function is mutilinear. 
\end{proof}


\subsection{Tensor and Wedge products}

\begin{definition}
Let $V_1:X\to\mathcal{L}^r,V_2:X\to\mathcal{L}^s$ be tensorfield. Then We define the tensorproduct $V_1\otimes V_2:X\to\mathcal{L}^{r+s}$ of them to be
\begin{align*}
 (V_1\otimes V_2)(p)(v_1,\cdots, v_r,v_{r+1},\cdots,v_{r+s}) = V_1(p)(v_1,\cdots,v_r)V_2(p)(v_{r+1},\cdots,v_{r+s})
\end{align*}
\end{definition}

\begin{proposition}
The operation $\bigotimes$ is bilinear and associative.
\end{proposition}

\begin{proof}
By substituting values, they are trivial.
\end{proof}

\begin{proposition}
Let $U\subset X$ be an open set and $V_1,\cdots,V_n\in\mathcal{V}^1(U)$ be a basis in $\mathcal{V}^1(U)$ then $\{\bigotimes_{j=1}^rV_{i_j}\}_{1\leq i_1,\cdots,i_r\leq r}$ is a basis in $\mathcal{V}^r(U)$.
\end{proposition}

\begin{proof}
Since $\otimes$ is bilinear, this is a tensor product thus the set in the statement is indeed a basis.
\end{proof}

\begin{definition}
Let $V\in\mathcal{V}^m(X)$ be a $m$-tensor. $V$ is said to be alternating if for any $p\in X$, $(v_1,\cdots,v_m)\in\bigoplus_{i=1}^m T_pX$ and $\sigma\in\mathfrak{S}_m$ we have
\begin{equation*}
V(p)(v_{\sigma(1)},\cdots,v_{\sigma(m)})=\sgn(\sigma)V(p)(v_1,\cdots,v_m)
\end{equation*}
Furthermore, such $V$ is called a $m$-form.
\end{definition}

\begin{notation}
The set of all $m$-forms is denoted by
\begin{equation*}
\mathcal{A}^m(X)=\{V\in\mathcal{V}^m(X)\:|\: V \text{ is a $m$-form.}\}
\end{equation*}
\end{notation}

\begin{definition}
Let $V_1\in\mathcal{A}^r(X),V_2\in\mathcal{A}^s(X)$ then the wedge product is 
\begin{equation*}
(V_1\wedge V_2)(p)(v_1,\cdots,v_{r+s}) = {\frac 1 {r!s!}}\sum_{\sigma\in\mathfrak{S}_{r+s}}\sgn(\sigma)V_1\otimes V_2(v_{\sigma(1)},\cdots,v_{\sigma(r+s)})
\end{equation*}
\end{definition}

\begin{proposition}
\label{sec:det_wedge}
Let $V_1,\cdots,V_n\in\mathcal{A}^1(X)$, $p\in X$ and $v_1,\cdots,v_n\in T_pX$ then we have
\begin{equation*}
(V_1\wedge\cdots\wedge V_n)(p)(v_1,\cdots,v_n) = \det(V_i(p)(v_j))_{i,j}
\end{equation*}
\end{proposition}

\begin{proof}
\begin{equation*}
(V_1\wedge\cdots\wedge V_n)(p)(v_1,\cdots,v_{n}) = {\frac 1 {1!\cdots 1!}}\sum_{\sigma\in\mathfrak{S}_{n}}\sgn(\sigma)\prod_{i=1}^nV_i(p)(v_\sigma(i))
\end{equation*}
\end{proof}

\begin{proposition}Similar to the case in tensorfields, we have the following statements.
\begin{enumerate}
\item $\mathcal{A}^m(X)$ is a subspace of $\mathcal{V}^m$ over $\mathbb{R}$.
\item $\mathcal{A}^m(X)$ is a module over $\mathcal{F}(X)$. 
\end{enumerate}
\end{proposition}

\begin{proof}
Trivial.
\end{proof}

\begin{proposition}
Let $V_1\in\mathcal{A}^r,V_2\in\mathcal{A}^s$, then $V_1\wedge V_2\in\mathcal{A}^{r+s}$ and such $\wedge:\mathcal{A}^r\times\mathcal{A}^s\to\mathcal{A}^{r+s}$ is bilinear.
\end{proposition}

\begin{proof}
Bilinearity follows from the bilinearity of $\otimes$. We will show that this is indeed well-defined.\\
\par Let $\sigma\in\mathfrak{S}_{r+s}$. Then we have
\begin{align*}
(V_1\wedge V_2)(p)(v_{\sigma(1)},\cdots,v_{\sigma(r+s)}) & = {\frac 1 {r!s!}}\sum_{\tau\in\mathfrak{S}_{r+s}}\sgn(\tau)V_1\otimes V_2(v_{\tau\circ\sigma(1)},\cdots,v_{\tau\circ\sigma(r+s)})\\
&=  \sgn(\sigma){\frac 1 {r!s!}}\sum_{\tau\circ\sigma\in\mathfrak{S}_{r+s}}\sgn(\tau\circ\sigma)V_1\otimes V_2(v_{\tau\circ\sigma(1)},\cdots,v_{\tau\circ\sigma(r+s)})\\
&=\sgn(\sigma)(V_1\wedge V_2)(p)(v_{\sigma(1)},\cdots,v_{\sigma(r+s)})
\end{align*}
\end{proof}

\begin{proposition}
\begin{equation*}
V_2\wedge V_1 = (-1)^{rs}(V_1\wedge V_2)
\end{equation*}
\end{proposition}
\begin{proof}
Let $\tau\in\mathfrak{S}_{r+s}$ to be such that
\begin{equation*}
\tau(i) = 
\begin{cases}
r+i \quad (1\leq i \leq s)\\
i-s\quad(s+1\leq i \leq r+s)
\end{cases}
\end{equation*}
Then clearly the inversion number is $N(\tau)=rs$. It is also obvious that
\begin{equation*}
V_2\wedge V_1(p)(v_{\tau(1)},\cdots,v_{\tau(r+s)}) = V_1\wedge V_2(p)(v_1,\cdots,v_{r+s})
\end{equation*}
\end{proof}

\begin{proposition}
Let $V_1\in\mathcal{A}^r,V_2\in\mathcal{A}^s,V_3\in\mathcal{A}^t$ then $(V_1\wedge V_2)\wedge V_3=V_1\wedge(V_2\wedge V_3)$.
\end{proposition}

\begin{proof}
\begin{align*}
(V_1\wedge V_2)\wedge V_3(p)(v_1,\cdots,v_{r+s+t})& = {\frac 1 {(r+s)!t!}}\sum_{\tau\in\mathfrak{S}_{r+s+t}}\sgn(\tau)(V_1\wedge V_2)\oplus V_3(v_{\tau(1)},\cdots,v_{\tau(r+s+t)})\\
&={\frac 1 {(r+s)!t!}}\sum_{\tau\in\mathfrak{S}_{r+s+t}}\sgn(\tau)\\
&({\frac 1 {r!s!}}\sum_{\sigma\in\mathfrak{S}_{r+s}}\sgn(\sigma)V_1\otimes V_2(v_{\tau\circ\sigma(1)},\cdots,v_{\tau\circ\sigma(r+s)}))\\
&V_3(v_{\sigma(r+s+1)},\cdots,v_{\sigma(r+s+t)})
\end{align*}
If for $\tau_1,\tau_2\in\mathfrak{S}_{r+s+t},\sigma_1,\sigma_2\in\mathfrak{S}_{r+s}$ we have $\tau_1\circ\sigma_1=\tau_2\circ\sigma_2$ then they satisfy the followings 
\begin{enumerate}[i.]
\item For any $r+s+1\leq i \leq r+s+t$ we have $\tau_1(i)=\tau_2(i)$.
\item From above we get $\tau_2^{-1}\circ\tau_1\in\mathfrak{S}_{r+s}$
\end{enumerate}
Fixing $\sigma_1$, there exists $(r+s)!$ many such $\sigma_2$. This implies that we can choose $\sigma_1$ to be the identity. Thus we get
\begin{align*}
(V_1\wedge V_2)\wedge V_3(p)(v_1,\cdots,v_{r+s+t})& = {\frac 1 {(r+s)!t!}}\sum_{\tau\in\mathfrak{S}_{r+s+t}}\sgn(\tau)(V_1\wedge V_2)\oplus V_3(v_{\tau(1)},\cdots,v_{\tau(r+s+t)})\\
&={\frac 1 {(r+s)!t!}}\sum_{\tau\in\mathfrak{S}_{r+s+t}}\sgn(\tau){\frac {(r+s)!} {r!s!}}V_1\oplus V_2\oplus V_3(v_{\tau(1)},\cdots,v_{\tau(r+s+t)})\\
& = {\frac 1 {r!s!t!}}\sum_{\tau\in\mathfrak{S}_{r+s+t}}\sgn(\tau)V_1\oplus V_2\oplus V_3(v_{\tau(1)},\cdots,v_{\tau(r+s+t)})\\
\end{align*}
From the previous proposition we get
\end{proof}

\section{Integration}

\begin{definition}
A differential $k$-form $\omega$ on a smooth manifold $M$ is a collection $\omega(p)\in A^k(T_pM)$ for all $p\in M$. 
\end{definition}

\begin{remark}
We can define what it means for $\omega$ to be continuous or smooth at some points $p\in M$ as follows.\\
\par First, we pick a chart $h:U\to V$ around $p$ and get the basis
\begin{equation*}
\{{\frac {\partial} {\partial x_1}},\cdots,{\frac {\partial} {\partial x_m}}\},
\end{equation*}
of $T_pM$ that moves with $p\in U$.\\
We also have a basis $A^k(T_pM)=\bigwedge^k(T_pM)^*$. Hence we can express $\omega$ as $p$ in terms of that basis and the scalars in this expression are functions on $U$.

\begin{equation*}
\omega(p) = \sum f_{i_1,\cdots,i_k}\cdot d_{x_{i_1}}\wedge\cdots\wedge d_{x_{i_k}}.
\end{equation*}
And we can require $ f_{i_1,\cdots,i_k}\cdots d_{x_{i_1}}$ to be smooth/continuous at $p$.
\end{remark}

\begin{example}
If $M=\mathbb{R}^n$, we have the canonical identification,
\begin{equation*}
T_pM = \mathbb{R}^n.
\end{equation*}
This gives us standard differential form of degree $n$. which is given by 
\begin{equation*}
e_1^*\wedge\cdots\wedge e^*_n,
\end{equation*}
where $e_1,\cdots,e_n$ is the standard basis of $\mathbb{R}^n$.
\end{example}

\begin{definition}
Let $f:M\to N$ be a smooth map of manifolds and $\omega$ be a differential form of degree $k$ on $N$. We define $f^*(\omega)$ of degree $k$ on $M$ by
\begin{equation*}
f^*(w)(p)(x_1,\cdots,x_k) = \omega(f(p))(\mathbf{d}f_p(x_1),\cdots,\mathbf{d}f_p(x_k)).
\end{equation*}
\end{definition}

\begin{definition}
A differential $n$-form $\omega$ on $M$ is said to be locally integrable if for any point $p\in M$, if for any point $p\in M$, there is one (hence any) chart $h:U\to V$ such that $\omega|_U=$
\end{definition}

\subsection{Group Theory}

\begin{definition}
Let $G$ be a group and $X$ be a set. An action of group $G$ on a set $X$ is a mapping $l:G\times X\to X$ such that for any $g,h\in G$ and $x\in X$
\begin{equation*}
l(gh,x) = m(g,m(h,x)).
\end{equation*}
\end{definition}

\begin{definition}
A stabilizer of an element $x\in X$ by a group action of $G$ is a subset of $G$ such that
\begin{equation*}
G_x=\{g\in G\:|\: gx=x\}.
\end{equation*}
\end{definition}

\begin{definition}
A group action is said to be free if for any $x\in X$ we have $G_x=\{1\}$.
\end{definition}

\subsection{Semi-Direct Product}

\begin{definition}
Let $G,\Gamma$ be groups and we have a group action induced by a group homomorphism $\rho:G\to\Aut(\Gamma)$ that is 
\begin{equation*}
(g,\gamma)\in G\times\Gamma,\quad g\gamma\mapsto \rho(g)(\gamma).
\end{equation*}
The semi-direct product of $G,\Gamma$ is a group
\begin{equation*}
\Gamma\rtimes G=( \Gamma\times G, \mu),
\end{equation*}
where $\mu$ is the multiplication defined as
\begin{equation*}
\mu((\gamma,g),(\delta,h)) = (\gamma\rho(g)(\delta),gh).
\end{equation*}
\end{definition}

\section{Lie groups}

\subsection{Manifolds and Submanifolds}

\begin{definition}
Let $f:X\to Y$ be a mapping between two topological spaces $X,Y$. $f$ is called a homeomorphism if 
\begin{enumerate}
\item $f$ is a bijection,
\item $f$ is continuous,
\item $f^{-1}$ is also continuous.
\end{enumerate}
\end{definition}

\begin{definition}
Let $U\subseteq\mathbb{R}^n,V\subseteq\mathbb{R}^m$ be open sets and $f:U\to V$ be a smooth map. Then the derivative of $f$ at $p\in U$ is 
\begin{equation*}
df(p) = \left({\frac {\partial f_i} {\partial x_j}}\right)_{ij}.
\end{equation*}
\end{definition}

\begin{proposition}
Let $f:U\to V, g:V\to W$ be smooth maps. Then for $p\in U$ we have
\begin{equation*}
d(g\circ f) = dg(f(p))df(p).
\end{equation*}
\end{proposition}

\begin{definition}
Let $U\subseteq\mathbb{R}^n,V\subseteq\mathbb{R}^m$ be open sets. A map $f:U\to V$ is called a diffeomorphism if 
\begin{enumerate}[i).]
\item $f$ is smooth. ($\Leftrightarrow$ arbitrary order of partial derivatives exists),
\item $f^{-1}$ is defined and is also a smooth map.
\end{enumerate}
\end{definition}

\begin{definition}
Let $X$ be a topological space. A chart on $X$ is a homeomorphism $h:U\to V$ where $U\subseteq X$ is open and $V\subseteq \mathbb{R}^n$ is open.
\end{definition}

\begin{definition}
An atlas $\mathcal{A}$ on a topological space $X$ is a collection of charts $\{h_\lambda\:|\: h_\lambda:U_\lambda\to V_\lambda\}_{\lambda\in \Lambda}$ such that
$\{U_\lambda\}_{\lambda\in\Lambda}$ is an open cover of $X$.
\end{definition}

\begin{definition}
An atlas $\mathcal{A}$ of $X$ is said to be smooth if for any two charts $h_1:U_1\to V_2,h_2:U_2\to V_2$. The following,
\begin{equation*}
h_2\circ h_1^{-1}:h_1(U_1\cap U_2)\to h_2(U_1\cap U_2),
\end{equation*}
is a smooth map. Such map is called a transition map.
\end{definition}

\begin{definition}
Let $X$ be a topological space and $\mathcal{A}_1,\mathcal{A}_2$ be smooth atlases. We say they are equivalent if $\mathcal{A}_1\cup\mathcal{A}_2$ is also smooth.
\end{definition}

\begin{proposition}
Above definition indeed defines an equivalence relation.
\end{proposition}
\begin{proof}
For any $h_1\in \mathcal{A}_1,h_2\in \mathcal{A}_2,h_3\in \mathcal{A}_3$, 
\begin{equation*}
h_3\circ h^{-1}_1 = h_3\circ h^{-1}_2\circ h_2\circ h^{-1}_1.
\end{equation*}
\end{proof}



\begin{definition}
A smooth manifold is a second countable Hausdorff topological space with equivalence classes of smooth atlases.
\end{definition}

\begin{definition}
Let $M,N$ be smooth manifolds, $f:M\to N$ be a map, and $p\in M$. $f$ is said to be smooth at $p$ if for one (hence any) pair of charts around $p$ and $f(p)$, 
\begin{equation*}
h_M:U_M\to V_M, h_N:U_N\to V_N,
\end{equation*}
the composed function 
\begin{equation*}
h_N\circ f \circ h_M^{-1}:V_M\to V_N
\end{equation*}
is smooth at $h_M(p)$.
\end{definition}

\begin{remark}
We can define a function $\dim:M\to N$ such that
\begin{equation*}
\dim(p)=\dim(V)_p,
\end{equation*}
for any chart $h:U\to V$ around $p$. And this function is locally constant. In particular, if $M$ is connected then it has a well-defined dimensions.
\end{remark}

\begin{definition}
Let $M,N$ be smooth manifold and $f:M\to N$ be a mapping which is smooth at $p\in M$. For any charts, 
\begin{equation*}
h_N\circ f \circ h_M^{-1}:V_M\to V_N,
\end{equation*}
the rank of $f$ at $p$ is such that
\begin{equation*}
\rk(f;p) = \rank(\mathbf{df}(h_M(p))(h_N\circ f\circ h^{-1}_M)).
\end{equation*}
\end{definition}

\begin{definition}
Let $M,N$ be smooth manifolds and $f:M\to N$ be a smooth map. A point $p$ is said to be regular with respect to the map f if it satisfies the equation,
\begin{equation*}
\rank(f,p) = \dim(f(p)).
\end{equation*}
And a point $q\in N$ is called a regular value if all $p\in f^{-1}(q)$ are regular.
\end{definition}

\begin{definition}
Let $M$ be a manifold. A subset $N\subseteq M $ is called an embedded submanifold if for any point $p\in N$, there is a chart $h_M:U_M\to V_M$ around $p$ such that
\begin{equation*}
h_M|_N:U_M\cap N\to V_M\cap \mathbb{R}^n,
\end{equation*}
 is a diffeomorphism where $n$ is the dimension of $N$.\\
\end{definition}

\begin{remark}
In particular, an embedded submanifold of an euclidean space is called a embedded manifold.
\end{remark}

\begin{theorem}
Let $f:M\to N$ be a smooth map between manifolds, and $q\in N$ be a regular value. Then $f^{-1}(q)\subset M$ is an embedded submanifold.
\end{theorem}

\begin{theorem}
Let $f:M\to N$ be a smooth map of manifolds $p\in M$ be a regular point, and $\dim(p) = \dim(f(p))$. Then $f$ is a local diffeomorphism of $p$. In other words, there is a neighborhood $U_M$ of $p$ in $M$ and $f(p)\in U_N\subset N$ such that
\begin{equation*}
f|_{U_M}:U_M\to U_N,
\end{equation*}
is a diffeomorphism. 
\end{theorem}

\subsection{Tangent Spaces}

\begin{definition}
Let $M\subseteq \mathbb{R}^n$ be an embedded manifold such that for some open set $U\subset\mathbb{R}^n$, there is $V\subset\mathbb{R}^n$ such that
\begin{equation*}
h:U\to V, \quad h_M:U\cap M\to V\cap \mathbb{R}^m,
\end{equation*}
is a diffeomorphism where $h_M$ is defined to be taking the first $m$ coordinate of the points in $V$. (Thus $m\leq n$).\\
\par The tangent space $T_pM$ of $M$ at $p$ is the subspace of $\mathbb{R}^n$ such that
\begin{equation*}
(\mathbf{dh}(p))^{-1}(\mathbb{R}^m)\subset\mathbb{R}^n.
\end{equation*}
\end{definition}

There are three definitions of tangent spaces and they are all equivalent. However, each of them has its own advantages. 

\begin{definition}[Coordinate tangent space]
Given a smooth manifold $M$ and a point $p\in M$. The coordinate tangent space of $p$ is such that
\begin{equation*}
T_p^{\mathbf{Coo}}M = \{(h,v)\:|\: h:U\to V\text{ is a chart}, v\in\mathbb{R}^m\}/\sim.
\end{equation*}
Where $\sim$ is an equivalence relation such that
\begin{equation*}
(h_1,v_1)\sim (h_2,v_2)\text{ if } (\mathbf{d}(h_2\circ h_1^{-1})(h_1(p)))(v_1) = v_2.
\end{equation*}
\end{definition}

\begin{definition}
Given a smooth manifold $M$, a point $p\in M$, and a smooth map $\alpha:I\to M$ whose domain $I$ is an open interval contains $0$. $\alpha$ is called a smooth curve if $\alpha(0)=p$.
\end{definition}

\begin{definition}
Two smooth curves $\alpha,\beta:I\to M$ through $p$ are said to be tangentially equivalent if for one (hence any) charts $h:U\to V$ around $p$, we have 
\begin{equation*}
d(h\circ\alpha)(0) = d(h\circ\beta)(0).
\end{equation*}
We denote such relation as $\sim_T$.
\end{definition}

\begin{definition}[Geometric tangent space]
The geometric tangent space at $p$ of a smooth manifold $M$ is such that
\begin{equation*}
T^{\mathbf{Geo}}_p=\{\alpha:I\to M\:|\: \alpha\text{ is a smooth curve}\}/\sim_T.
\end{equation*}
\end{definition}

\begin{definition}
A germ of smooth functions of manifolds $M$ at $p$ is an equivalence class of tuples $(U,f)$ where
\begin{enumerate}[i).]
\item $U\subset M$ is a neighborhood of $p$,
\item $f:U\to\mathbb{R}$ is smooth,
\end{enumerate}
and two tuples $(U_1,f_1),(U_2,f_2)$ are equivalent if there is a neighborhood $V$ of $p$ such that $V\in U_1\cap U_2$ and $f_1|_V=f_2|_V$. \\
\par And we denote the set of germs at $p$ as
\begin{equation*}
\mathcal{C}^\infty(p).
\end{equation*}
\end{definition}

\begin{remark}
$\mathcal{C}^\infty(U,\mathbb{R})$ and $\mathcal{C}^\infty(p)$ are rings, in fact $\mathbb{R}$-algebras. 
\end{remark}

\begin{definition}
Let $R$ be a ring and $A$ be a bimodule over $R$. A $R$-derivation in $A$ is an operator $X:A\to A$ such that the Leibniz rule holds. In other words, 
\begin{equation*}
X(ab) = aX(b)+X(a)b,
\end{equation*}
holds for all $a,b\in A$.
\end{definition}

\begin{definition}[Algebraic tangent space]
The algebraic tangent space $T^{\mathbf{Alg}}_pM$ of $M$ at $p$ is the set of $\mathbb{R}$-derivations $X:\mathcal{C}^\infty(p)\to\mathbb{R}$. 
\end{definition}

\begin{remark}
In the above definition, $\mathbb{R}$ is considered as a $\mathcal{C}^\infty(p)$-bimodule via the evaluation map $f\mapsto f(p)$.
\end{remark}

\begin{theorem}
The following are isomorphisms of $\mathcal{R}$-vector spaces.
\begin{align*}
T^{\mathbf{Geo}}_pM&\to T^{\mathbf{Alg}}_pM, \alpha\mapsto(f\mapsto (f\circ\alpha)'(0)),\\
T^{\mathbf{Alg}}_pM&\to T^{\mathbf{Coo}}_pM, X\mapsto (h, ((Xh_i)(p))_{i=1,\cdots,n}),\\
T^{\mathbf{Coo}}_pM&\to T^{\mathbf{Geo}}_pM, (h,v)\mapsto \alpha(t)=h^{-1}(h(p)+t\cdot v).
\end{align*}
\end{theorem}

\begin{proposition}
$\mathcal{C}^\infty(p)$ is a local ring with its maximal ideal
\begin{equation*}
\mathfrak{m}_p=\{f\in\mathcal{C}^\infty(p)\:|\: f(p) = 0\}.
\end{equation*}
Moreover, if we have a derivation $X:\mathcal{C}^\infty(p)\to\mathbb{R}$, the restricted derivation $X|_{\mathfrak{m}_p}$ is in
$\Hom_{\mathbb{R}}(\mathfrak{m}_p/\mathfrak{m}_p^2)$. 
And by this restriction, we get an isomorphism between $T^{\mathbf{Alg}}_pM$ and $\Hom_{\mathbb{R}}(\mathfrak{m}_p/\mathfrak{m}_p^2,\mathbb{R})$.
\end{proposition} 

\begin{remark}
In this way, a smooth manifold is recognized as a locally ringed space, locally isomorphic to $\mathbb{R}^n$.
\end{remark}

\begin{lemma}
Let $V$ be a finite dimensional $\mathbb{R}$-vector space. It has a tautological smooth manifold structure by taking charts such 
that the sets of isomorphisms of $V$ and $\mathbb{R}^n$ given by arbitrary basis of $V$.\\
\par Let $V$ be a finite dimensional $\mathbb{R}$-vector space
\par We claim that we have canonical isomorphisms
\begin{equation*}
T_pV\to V,
\end{equation*}
for any $p\in V$,
\begin{align*}
V\to T_p^{\mathbf{Coo}}V,& v\mapsto(h,h(v)),\\
V\to T_p^{\mathbf{Geo}}V,& v\mapsto(t\mapsto p+tv),\\
V\to T_p^{\mathbf{Alg}}V,& v\mapsto\left(f\mapsto{\frac d {dt}}\bigg{|}_{t=0}f(p+tv)\right)
\end{align*}
\end{lemma}

\begin{definition}
Let $f:M\to N$ be a map of smooth manifolds which is smooth at $p\in M$. Its differential of $p$ is the linear map
\begin{equation*}
\mathbf{d}f(p) = \mathbf{d}_p(f):T_pM\to T_{f(p)}N,
\end{equation*}
defined as follows.
\begin{enumerate}[1).]
\item Geometric tangent space : $\mathbf{d}_p(f)(\alpha) = f\circ\alpha$ where $\alpha$ is a smooth curve.
\item Algebraic tangent space : $\mathbf{d}_p(f)(X)(\varphi) = X(\varphi\circ f)$ where $\varphi\in\mathcal{C}^\infty(f(p))$.
\item Coordinate tangent space : $\mathbf{d}_p(f)(h_M,v_M) = (h_N,d_{h_M(p)}(h_N)$.
\end{enumerate}
\end{definition}

\begin{remark}
Given a chart $h:U\to V$ around $p\in M$. $h$ consists of coordinate functions $h_i$ where $1\leq i\leq m$ for $V\subset\mathbb{R}^m$. We have for each $i$
\begin{equation*}
\mathbf{d}_ph_i:T_pM\to\mathbb{R},
\end{equation*}
and 
\begin{equation*}
B=\{d_ph_1,\cdots,d_ph_m\}
\end{equation*}
is a basis of the dual space $(T_pM)^*$.\\
\par Let 
\begin{equation*}
\{{\frac {\partial} {\partial x_1}},\cdots,{\frac {\partial} {\partial x_m}}\}
\end{equation*}
be the dual basis of $B$. By definition, this means that for any $1\leq i,j\leq m$, we have
\begin{equation*}
{\frac {\partial} {\partial x_i}}h_j = d_ph_j({\frac {\partial} {\partial x_i}}) = \delta_{ij}.
\end{equation*}
\end{remark}

\begin{proposition}
Let $f:M\to N$ be a map between smooth manifolds which is smooth and $q\in N$ be a regular value. For $p\in f^{-1}(q)$, we have
\begin{equation*}
T_pf^{-1}(q) = \mathbf{d}_p(f)^{-1}(0)\subset T_pM.
\end{equation*}
\end{proposition}

\begin{proof}

\end{proof}

\subsection{Immersions and Submersions}

\begin{definition}
Let $f:M\to N$ be a smooth map of smooth manifolds. $f$ is called an
\begin{enumerate}[1).]
\item immersion if $\mathbf{d}_pf:T_pM\to T_{f(p)}N$ is injective for any $p\in M$,
\item submersion, if $\mathbf{d}_pf:T_pM\to T_{f(p)}N$ is surjective for any $p\in M$.
\end{enumerate}
\end{definition}

\begin{remark}
An immersion need not be injective. The counter example is 
\begin{equation*}
e^{ix}:\mathbb{R}\to S^1,
\end{equation*}
is an immersion.
\end{remark}

\begin{remark}
A submersion need not be injective. The counter example is 
\begin{equation*}
i_U:U\to M,
\end{equation*}
an inclusion map is a submersion.
\end{remark}

\begin{remark}
We know that if $f$ is a submersion, then $f^{-1}(q)$ is an embedded submanifold. However, if $f$ is an immersion, even it is injective,  $f(M)$ need not be an embedded submanifold of $N$.
\end{remark}

\begin{definition}
An immersed submanifold is an image of an injective immersion. 
\end{definition}

\begin{remark}
We endow $f(M)$ with the transported topology and differential structure from $M$ so that $f$ becomes a diffeomorphism between $M$ and $f(M)$. But this topology need not be the relative topology from $N$. It may be strictly finite.
\end{remark}

\begin{example}
Let $T=S^1\times S^1$ be a torus. Let $r\in\mathbb{R}$. We consider a map $f:\mathbb{R}\to T$ such that
\begin{equation*}
f(x) = (e^{2\pi t x},e^{2\pi ri x}).
\end{equation*}
This is an immersion for any $r$. We examine this by several cases. \\
\par First, when $r$ is not a rational number then $f$ is injective, the image is an immersed manifold. However, a copy of $\mathbb{R}$. But this image is a dense subset of the torus. \\
\par Second, if $r$ is rational then $f$ is not injective. It is going to factor through an injective immersion $\mathbb{R}/b\mathbb{Z}\to T$ where $r={\frac a b}$, $a,b\in\mathbb{Z}$ are coprime. This image is not only immersed but also embedded. \\
\end{example}

\begin{remark}
If $f:M\to N$ is an immersion, $\mathbf{d}f_(p)$ identifies $T_pM$ with a linear subspace of $T_{f(p)}N$. 
\end{remark}

\begin{proposition}
If $f:M\to N$ is an injective immersion, that is also closed, then its image is an embedded submanifold. 
\end{proposition}

\begin{remark}
Thus we have the notion of a closed submanifold. 
\end{remark}


\section{Basic Lie Theory}

\subsection{Lie Groups}

\begin{definition}
A Lie group is a quintuple $(G,e,\mu,\iota)$ such that $\mu,\iota$ are smooth and $G$ forms a group under the multiplication $\mu:G\times G\to G$. Furthermore, its identity is $e$ and $\iota:G\to G$ is the inversion.  
\end{definition}

\begin{remark}
There are some redundancies in the above definition of Lie groups. In particular, it is enough for us to assume that the multiplication is smooth and make $G$ into a group. 
%Give a proof that this is actually enough, (ie, the inversion is naturally define and smooth). 
\end{remark}

\begin{definition}
The general linear group of degree $n$ over a field $\mathbb{F}$ is a group $\GL_n(\mathbb{F})$ such that 
\begin{equation*}
\GL_n(\mathbb{F}) = \{A\in\mathbb{F}^{n\times n}\:|\: \det A \not = 0\},
\end{equation*}
together with the matrix multiplication defines the multiplication of the group.
\end{definition}

\begin{definition}
The special linear group of degree $n$ over a field $\mathbb{F}$ is a subgroup $\SL_n(\mathbb{F})$ of $\GL_n(\mathbb{F})$ such that 
\begin{equation*}
\SL_n(\mathbb{F}) = \{A\in\GL_n(\mathbb{F})\:|\: \det A = 1\}.
\end{equation*}
\end{definition}

\begin{example}
% Check these statement, consider putting them in the section where we discuss compact lie groups.
Here we list some notable examples
\begin{enumerate}[1).]
\item $(\mathbb{R},0,+,-)$ is a non-compact connected Lie group.
\item $(\mathbb{R}^\times,1,\cdot,(\cdot)^{-1})$ is a non-compact, non-connected Lie group.
\item $(\GL_n(\mathbb{R}),I_n,\cdot,(\cdot)^{-1})$ is a non-compact, non-connected Lie group. And we have $T_pM = \End(\mathbb{R^n})=\mathbb{R}^{n\times n}$.
\end{enumerate}
\end{example}
\begin{example}
The circle in the complex plane
\begin{equation*}
S^1 =\{z\in\mathbb{C}\:|\: \vert z\vert = 1\},
\end{equation*}
forms a compact connected Lie group $(S^,1,\cdot,(\cdot)^{-1})$.\\
\par In order to see this, we let $z = x+iy$ for $x,y\in\mathbb{R}$, 
\begin{equation*}
f(z) = f(x,y) = x^2+y^2.
\end{equation*}
Then its derivative is 
\begin{equation*}
df_{(x,y)} = (2x,2y).
\end{equation*}
This is singular only at the origin of the plane which does not belong to $S^1$.\\
\par We can compute 
% Check why the tangent space is the kernel of the derivative.
\begin{equation*}
T_{(x_0,y_0)}S^1 = \Ker(df_{(x_0,y_0)}) = \{(x,y)\in\mathbb{R}^2\:|\: x_0x+y_0y = 0\}.
\end{equation*}
At $(1,0)$ we get
\begin{equation*}
T_{(1,0)}S^1 = i\mathbb{R}.
\end{equation*}
\end{example}

\begin{example}
$(\SL_n(\mathbb{R}),I_n,\cdot,(\cdot)^{-1})$ is a non-compact Lie group which is an embedded submanifold of $\GL_n(\mathbb{R})$. This can be seen by
\begin{equation*}
\SL_n(\mathbb{R}) = \det^{-1}(I_n),
\end{equation*}
where $\det:\GL_n(\mathbb{R})\to\mathbb{R}^\times$ is a determinant.
%check the regularity and immersion/submersion.
\end{example}

\subsection{Important Examples of Lie Groups}



\begin{lemma}
$\det:R^{n\times n}\to \mathbb{R}$ is differentiable at any $A\in\GL_n(\mathbb{R})$ and its derivative at $A$ is
\begin{equation*}
d_A\det:\mathbb{R}^{n\times n}\to\mathbb{R},\quad H\mapsto \det(A)\tr(A^{-1}H).
\end{equation*}
\end{lemma}

\begin{proof}
First we consider the case where $A=I_n$. \\
\par Let $H = (\underline{h}_1,\cdots,\underline{h}_n)$ where $\underline{h}_1,\cdots,\underline{h}_n\in\mathbb{R}^n$. Then the directional derivative of $\det$ at $A$ with respect to $H$ is computed as 
\begin{align*}
\det(I_n+tH) &= \det(\underline{e}_1+t\underline{h}_1,\cdots,\underline{e}_n+t\underline{h}_n),\\
& = \sgn\sum_{\sigma\in\mathfrak{S}_n}\prod_{i=1}^n(\delta_{i\sigma(i)}+t(\underline{h}_i)_{\sigma(i)}),\\
& = \det I_n+\sum_{k=1}^n\sgn\sum_{\sigma\in\mathfrak{S}_n}t(\underline{h}_k)_{\sigma(k)}\prod_{i=1,i\not=k}^n(\delta_{i\sigma(i)}+t(\underline{h}_i)_{\sigma(i)}),\\
& = \det I_n + \sum_{i=1}^n th_{ii}+o(t),\\
& = \det I_n+t\tr H+o(t).
\end{align*}
By formally differentiating $\det$ we have
\begin{equation*}
\lim_{t\to 0}{\frac {\det I_n + t\tr H+o(t)-\det I_n} {t}} = \tr H.
\end{equation*}
For the general case, observe that 
\begin{equation*}
\det(A+H)-\det H = \det A(\det(A^{-1}+A^{-1}H)-\det (A^{-1}H).
\end{equation*}
%Prove det is smooth thus for small enough t, I+tH is still an automorphism
\end{proof}

\begin{corollary}
$1\in\mathbb{R}$ is a regular value of $\det:\GL_n(\mathbb{R})\to\mathbb{R}^\times$.
\end{corollary}

\begin{proof}
Let $A\in\GL_n(\mathbb{R})$ such that $\det A =1$. Then from the previous lemma, we have
\begin{equation*}
d_A\det(H) = \tr H.
\end{equation*}
The trace function is a non-degenerate symmetric bilinear form by %Put the proof of this in the linear algebra section.
We conclude that $d_A\det$ has a full-rank at $A$.
\end{proof}

\begin{definition}
The tangent space $\mathfrak{sl}_n(\mathbb{F})$ of $\SL_n(\mathbb{F})$ at $I_n$ is called the special linear Lie algebra which has the following closed form,
\begin{equation*}
\mathfrak{sl}_n(\mathbb{F}) = \{H\in\mathbb{R}^{n\times n}\:|\: \tr(A^{-1}H) = 0\}.
\end{equation*}
\end{definition}

\begin{definition}
Let $V$ be a finite dimensional $\mathbb{R}$-vector space and $\langle\cdot,\cdot\rangle_V:V\times V\to\mathbb{R}$ be a a symmetric bilinear form. We define
\begin{equation*}
O(V,\langle\cdot,\cdot\rangle) = \{A\in\Aut(V)\:|\: \forall \underline{x},\underline{y}\in V,\quad \langle A\underline{x},A\underline{y}\rangle_V=\langle \underline{x},\underline{y}\rangle_V\},
\end{equation*} 
which is a subgroup of $\Aut(V)$.
\end{definition}

\begin{remark}
$O(V,\langle\cdot,\cdot\rangle_V)$ above is a disconnected topological space.
%Give a proof to this statement.
\end{remark}

\begin{remark}
%Add Sylvester's theorem in the linear algebra section with a proof.
By Sylvester's theorem%reference the theorem
, such group $O(V,\langle\cdot,\cdot\rangle_V)$ is isometric % give a definition of isometric in linear algebra section
to some $(\mathbb{R}^n,\beta_{p,q})$ where $p,q\in\mathbb{N}\cup\{0\},p+q=n$ and 
\begin{equation*}
\beta_{p,q}(\underline{x},\underline{y}) = \sum_{i=1}^p x_iy_i-\sum_{i=p+1}^{p+q}x_iy_i.
\end{equation*}
\end{remark}

\begin{notation}
\begin{equation*}
O(p,q) = O(\mathbb{R}^{p+q},\beta_{p,q}).
\end{equation*}
\end{notation}

\begin{lemma}
The map
\begin{equation*}
\Phi:\mathbb{R}^{n\times n}\to S^2(\mathbb{R}^{n})^*,\quad \Phi(A) = \langle A\cdot,A\cdot\rangle,
\end{equation*}
is differentiable at any $A\in\mathbb{R}^{n\times n}$ and submersive at any $A\in O(p,q)$ where $p+q=n$.
\label{symmetric_bilinear_space_lie_group}
\end{lemma}
\begin{proof}
Let $A,H\in\mathbb{R}^{n\times n}$ and $\underline{x},\underline{y}\in\mathbb{R}^n$. We first compute $\Phi(A+tH)(\underline{x},\underline{y})$ which is 
\begin{align*}
\langle(A+tH)\underline{x},(A+tH)\underline{y}\rangle & = \langle A\underline{x},A\underline{y}\rangle+\langle A\underline{x},tH\underline{y}\rangle+\langle tH\underline{x},A\underline{y}\rangle+\langle tH\underline{x},tH\underline{y}\rangle,\\
& = \langle A\underline{x},A\underline{y}\rangle+\langle A\underline{x},tH\underline{y}\rangle+\langle tH\underline{x},A\underline{y}\rangle+o(t).
\end{align*}
Therefore, by formally differentiation $\Phi$ at $A$ with respect to $H$, we have
\begin{equation*}
\lim_{t\to0}{\frac {\Phi(A+tH)(\underline{x},\underline{y})-\Phi(A)(\underline{x},\underline{y})} {t}} = \langle A\underline{x},H\underline{y}\rangle+\langle H\underline{x},A\underline{y}\rangle.
\end{equation*}
In order to show that $d\Phi_A$ is surjective for all $A\in O(p,q)$, we need to show that for any $(\cdot,\cdot)\in S^2(\mathbb{R}^n)^*$, there is $H\in\mathbb{R}^{n\times n}$ such that 
\begin{equation*}
(\underline{x},\underline{y}) = \langle A\underline{x},H\underline{y}\rangle+\langle H\underline{x},A\underline{y}\rangle.
\end{equation*}
We first show this when $A=I_n$. From Lemma %state the proposition, where any symmetric bilinear form has a symmetric matrix representing it.
, there is a symmetric matrix $B\in\mathbb{R}^{n\times n}$ such that 
\begin{equation*}
(\underline{x},\underline{y}) = \underline{x}^TB\underline{y}.
\end{equation*}
Let $H= {\frac 1 2}B$, we proved that $d\Phi_{I_n}$ is surjective. For the general case, for any $A\in O(p,q)$, consider the symmetric bilinear form
\begin{equation*}
(A^{-1}\cdot,A^{-1}\cdot).
\end{equation*}
By the previous argument, there is $H'\in\mathbb{R}^{n\times n}$ such that 
\begin{equation*}
(A^{-1}\underline{x},A^{-1}\underline{y}) = \langle\underline{x},H'\underline{y}\rangle+\langle H'\underline{x},\underline{y}\rangle.
\end{equation*}
Substitute $A\underline{x},A\underline{y}$ for $\underline{x},\underline{y}$, we derive the statement.
\end{proof}

\begin{corollary}
$O(p,q)$ is a Lie gropu of dimension ${n \choose 2}$. Furthermore, if $p=0$ or $q=0$, then it is compact. 
\end{corollary}

\begin{proof}
By Lemma \ref{symmetric_bilinear_space_lie_group}, we have $O(p,q)$ is a Lie group. \\
%Complete the proof.
\par For its dimension, observe that 
\begin{equation*}
\dim S^2(\mathbb{R}^n)^* = {\frac {(n+1)n} 2}.
\end{equation*}
Therefore we have
\begin{equation*}
\dim O(p,q) = n^2 -\dim S^2(\mathbb{R}^n)^* = {n\choose 2}.
\end{equation*}
%Prove the closedness of O(p,q)
For the compactness, notice that $O(p,q)$ is closed. Let us now consider the operator norm on $O(p,q)$, then we have for any $A\in O(p,q)$,
\begin{equation*}
{\frac {\langle A\underline{x},A\underline{x}\rangle} {\langle\underline{x},\underline{x}\rangle}} = 1.
\end{equation*}
Thus $\Vert A\Vert = 1$. By Heine-Borel theorem, $O(p,q)$ is compact.
\end{proof}

\begin{proposition}
%Put the notation of I_{p,q}
Let $F\mathbb{R}^{n\times n}\to \mathbb{R}^{n\times n}$ such that 
\begin{equation*}
F(A) = A^TI_{p,q}AI_{p,q}^{-1}.
\end{equation*}
Then $O(p,q) = F^{-1}(I_n)$ and thus we have
\begin{equation*}
T_{I_n}O(p,q) = \{ H\in\mathbb{R}^{n\times n}\:|\: H^T=-I_{p,q}HI_{p,q}^{-1}\}.
\end{equation*}
\end{proposition}

\begin{proof}
Let $A\in \mathbb{R}^{n\times n}$, then by the definition
\begin{equation*}
A\in O(p,q)\Leftrightarrow A^TI_{p,q}A = I_{p,q}.
\end{equation*}

This shows the first statement that $O(p,q)=F^{-1}(I_n)$. For the second statement. we have
\begin{equation*}
(I_n+tH)^TI_{p,q}(I_n+tH)I_{p,q}^{-1} = I_n+tH^T+I_{p,q}tHI_{p,q}^{-1}+t^2H^TH = I_n+tH^T+tH+o(t).
\end{equation*}
Formally differentiating $F$ at $I_n$ with respect to $H$, we get
\begin{equation*}
\lim_{t\to 0}{\frac {(I_n+tH)^TI_{p,q}(I_n+tH)I_{p,q}^{-1} -I_n} {t}} = H^T+I_{p,q}HI_{p,q}^{-1}.
\end{equation*}
\end{proof}

\begin{notation}
\begin{equation*}
T_{I_n}O(p,q) = \mathfrak{o}(p,q).
\end{equation*}
\end{notation}

\begin{remark}
In particular, for $p=0$, we have
\begin{equation*}
\mathfrak{o}(n,0)=\mathfrak{o}(0,n) = \{ H\in\mathbb{R}^{n\times n}\:|\: H^T=-H\}.
\end{equation*}
\end{remark}

\begin{example}
The isometry group of the $n$-dimensional Euclidean space is 
\begin{equation*}
\mathbb{R}^n\rtimes O(n,0).
\end{equation*}
%Define the semiproduct in the group section
\end{example}

\begin{example}
The Minkowski $4$-space is  the vector space $\mathbb{R}^4$ equipped with $\beta_{3,1}$. Its linear automorphism group is
\begin{equation*}
O(3,1)
\end{equation*}
which is a non-compact Lie group called the Lorenz group.\\
\par The group of affine isometries of the Minkowski $4$-space is 
\begin{equation*}
\mathbb{R}^4\rtimes O(3,1)
\end{equation*}
which is called the Poincare group.
\end{example}

\begin{proposition}
For a Riemannian manifold $M$, its isometry group $\Isom(M)$ is a Lie group.
\end{proposition}

\begin{proof}
%give a proof
\end{proof}

\begin{example}
The isometry group of $S^n$ is $O(n,0)$.
%Give proof
\end{example}

\begin{example}
%Give the definition of the Poincare half-plane model
For the Poincare half-plane model of the Hyperbolic geometry $\mathbb{H}$ endowed with metric
\begin{equation*}
{\frac {dx\wedge dy} {y^2}}.
\end{equation*}
Let us consider the action of $\SL_2(\mathbb{R})$ on $\mathbb{H}$ by 
\begin{equation*}
\begin{pmatrix}a&b\\ c&d\end{pmatrix}\in\SL_2(\mathbb{R}),z\in\mathbb{H},\quad \begin{pmatrix}a&b\\ c&d\end{pmatrix}z\mapsto {\frac {az+b} {cz+d}}.
\end{equation*}
%Revise the statement of the factoring.
This action is going to factor through $\SL_2(\mathbb{R})/\{\pm I_2\}$ and identifies this group with the group of orientation preserving isometries $\Isom^+(\mathbb{H})$. Furthermore, for any $A\in\Isom(\mathbb{H})\backslash\Isom^+(\mathbb{H})$, there is $B\in\Isom^+(\mathbb{H})$ such that 
\begin{equation*}
A = (z\mapsto -(\overline{z})^{-1})\circ B.
\end{equation*}
%Give the matrix form of the (z\mapsto -(\overline{z})^{-1})
\end{example}

\subsection{Important homomorphisms and their properties.}

Recall if $f:M\to N$ is a smooth map of smooth manifolds and $p\in M$, we get $df(p):T_pM\to T_{f(p)}N$ is linear. 

\begin{lemma}
For any manifolds $M,N$, there exists a unique manifold with its underlying space $M\times N$ such that projections
\begin{equation*}
\pi_1:M\times N\to M,\quad \pi_2:M\times N\to N
\end{equation*}
are both smooth. Furthermore, it satisfies the following universal property.\\
\par For any manifold $L$ and a map $f:L\to M\times N$. 
\begin{equation*}
f \text{ is smooth}\Leftrightarrow f\circ\pi_1,f\circ\pi_2\text{ are smooth}.
\end{equation*}
\label{universal_property_Lie_group_product}
\end{lemma}

\begin{proof}
%Give a proof.
\end{proof}

\begin{lemma}
Let $M,N$ be manifolds and $p\in M,q\in N$ be points. Then there exists an isomorphism 
\begin{equation*}
T_pM\times T_qN\cong T_{(p,q)}M\times N,
\end{equation*}
which is induced by projections and given by the formula
\begin{equation*}
d_p(\id,i_q)+d_q(i_p,\id)T_pM\times T_qN\to T_{(p,q)}M\times N.
\end{equation*}
where $i_p:\{p\}\to M,i_q:\{q\}\to N$ are inclusions.
\label{isomorphism_product_tangent_space}
\end{lemma}

\begin{proof}
%Give a proof.
\end{proof}

\begin{proposition}
Let $(G,\mu,\iota,1)$ be a Lie group and $\mathfrak{g}=T_1G$. We have
\begin{align*}
d\mu_{(1,1)}:\mathfrak{g}\times\mathfrak{g}\to\mathfrak{g}&, (X,Y)\mapsto X+Y.\\
d\iota(1):\mathfrak{g}\to\mathfrak{g}&, X\mapsto -X
\end{align*}
\end{proposition}

\begin{proof}
By Lemma \ref{isomorphism_product_tangent_space}, we have
\begin{equation*}
T_1G\times T_1G\cong T_{(1,1)}G\times G,
\end{equation*}
which is given by
\begin{equation*}
d_1(\id,i_1)+d_1(i_1,\id).
\end{equation*}
We also have
\begin{equation*}
\mu(\id,i_1)(g,1) =\mu(i_1,\id)(1,g) = g=\id(g).
\end{equation*}
Using the above equation and the product rule of differentiation, we get
\begin{equation*}
X=d_{(1,1)}\mu(\id,i_1)(X) = d_{(1,1)}\mu\circ d_1(\id,i_1).
\end{equation*}
Similarly for $\mu(i_1,\id)$. By the linearity of the derivatives we get 
\begin{equation*}
d_{(1,1)}\mu(X,Y) = X+Y.
\end{equation*}
For $\iota$, we have that 
\begin{equation*}
\varphi:G\times G\to G, \quad F(g,h) = (g,gh)
\end{equation*}
is smooth since
\begin{equation*}
\pi_1\circ \varphi= \id_G,\quad \pi_2\circ \varphi = \mu
\end{equation*}
are both smooth. (By Lemma \ref{universal_property_Lie_group_product}). Differentiating $\varphi$ at $(1,1)$, we get
\begin{align*}
d_{(1,1)}\varphi = 
\end{align*}
Now let us define a function 
\begin{equation*}
\psi:G\times G\to G\times G,\quad \psi(g,h) = (g,gh^{-1}).
\end{equation*}
%Complete the proof.
\end{proof}

\begin{definition}
A Lie group homomorphism is a smooth map of Lie groups that is a homomorphism of groups.
\end{definition}

\begin{remark}
If $f:G\to H$ is a Lie group homomorphism then 
\begin{equation*}
df(1):\mathfrak{g}\to \mathfrak{h}
\end{equation*}
is a linear map.
%Consider if this can be put in somewhere else.
\end{remark}

\begin{definition}
Let $G$ be a Lie group. The adjoint action of $G$ on itself is 
\begin{equation*}
\underline{\Ad}(g):G\to G, h\mapsto ghg^{-1}
\end{equation*}
which is a group homomorphism.
\end{definition}

\begin{definition}
Let $G$ be a Lie group and $\mathfrak{g} = T_1G$. Then we define
\begin{equation*}
\Ad(g) = d_1\underline{\Ad}(g):\mathfrak{g}\to\mathfrak{g}.
\end{equation*}
We call this the adjoint action of $G$ on $\mathfrak{g}$.
\end{definition}

\begin{remark}
The term, "action" in the definition above is justified by the chain rule
\begin{equation*}
\Ad(g\cdot h) = \Ad(g)\circ\Ad(h).
\end{equation*}
This follows from that $\underline{\Ad}(g)(1) = 1$. 
\end{remark}

\begin{remark}
$\Ad(g)$ is invertible since for any $X\in\mathfrak{g}$, since by the definition of derivatives
\begin{equation*}
\id = \Ad(1) = \Ad(g)\circ\Ad(g^{-1}).
\end{equation*}
\label{invertible_ad_action}
\end{remark}

\begin{definition}
Let $G$ be a Lie group and $\mathfrak{g}=T_1G$. By regarding the adjoint action $\Ad$ as a function 
\begin{equation*}
\Ad(\cdot):G\to\GL(\mathfrak{g}),\quad g\mapsto (\Ad(g)(\cdot):\mathfrak{g}\to\mathfrak{g}),
\end{equation*}
by Remark \ref{invertible_ad_action}, we now define the adjoint action of $\mathfrak{g}$ on itself such that 
\begin{equation*}
\ad:\mathfrak{g}\to\End(\mathfrak{g}),\quad X\mapsto d_1\Ad(X).
\end{equation*}
\end{definition}

\begin{definition}
Let $G$ be a Lie group and $\mathfrak{g}=T_1G$. The Lie bracket is $[\cdot|\cdot]:\mathfrak{g}\times\mathfrak{g}\to\mathfrak{g}$ such that 
\begin{equation*}
[X|Y] = \ad(X)(Y).
\end{equation*}
\end{definition}

\begin{proposition}
Let $G=\GL_n(\mathbb{R})$ and $\mathfrak{g}=\mathbb{R}^{n\times n}$. Let $g\in G$ and $X,Y\in\mathfrak{g}$. We have
\begin{equation*}
[X|Y] = XY-YX.
\end{equation*}
\end{proposition}

\begin{proof} Let $g\in G$, 
\begin{align*}
\Ad(g)X &= d\underline{\Ad}(g)(1)X,\\
& = g(1+X)g^{-1}-g1g^{-1}\mod{o(X)},\\
& = gXg^{-1}\mod{o(X)},\\
&=gXg^{-1}.
\end{align*}
In particular $\underline{Ad}$ is a liner map. Now we compute the Lie bracket
\begin{equation*}
[X|Y] = \ad(X)(Y) = [E_Y\circ\ad](X),
\end{equation*}
where $E_Y$ is the evaluation map 
\begin{equation*}
E_Y:\End(\mathfrak{g})\to\mathfrak{g}, \phi\mapsto\phi(Y).
\end{equation*}
\begin{align*}
[X|Y] & = [E_Y\circ\ad](X),\\
& = d[g\mapsto\Ad(g)Y](1)(X),\\
&=d[g\mapsto gYg^{-1}](1)(X).
\end{align*}
By the first computation we did, we see that
\begin{align*}
[X|Y] = (1+X)Y(1+X)^{-1}-Y\mod{o(X)}.
\end{align*}
We have the following identity
\begin{equation*}
(1-X)^{-1}=1+X+X^2+\cdots.
\end{equation*}
Substituting $-X$ we derive that
\begin{equation*}
1+X=\sum_{i=0}^\infty (-1)^iX^i.
\end{equation*}
And we only need at most degree $1$ terms of $X$. We conclude that
\begin{equation*}
[X|Y] =XY-YX.
\end{equation*}
\end{proof}

\begin{remark}
This works for any matrix groups such as $\SL_n(\mathbb{R}),O(p,q)$.
\end{remark}

\begin{proposition}
Let $f:G\to H$ be a Lie group homomorphism. For $g\in G$, we have
\begin{equation}
df(1)\circ\Ad(g)=\Ad(f(g))\circ df(1).
\label{nat_transform_gh}
\end{equation}
And for $X,Y\in\mathfrak{g}$, we have
\begin{equation}
df(1)([X|Y]_G)=[df(1)X,df(1)Y]_H.
\end{equation}
\label{comp_homo}
\end{proposition}

\begin{proof}
Let us consider the composition of $f$ and $\underline{\Ad}$. By definition, we see
\begin{equation*}
f\circ\underline{\Ad}(g)(h) =  f(g)f(h)f(g)^{-1}=\underline{\Ad}(f(g))(f(h)).
\end{equation*}
Since $f(1)=1,\underline{\Ad}(1) = 1$ and by the chain rule we have Equation \eqref{nat_transform_gh}.\\
\par For the second equation, we first note that $\Ad\circ f:G\to \GL(\mathfrak{h})$ and 
\begin{equation*}
[d_1fX,d_1fY]_H = d_1(\Ad\circ f)(X)(Y).
\end{equation*}

\end{proof}

\subsection{Lie Algebras}

\begin{definition}
A Lie algebra a (finite dimensional) vector space $L$ over $\mathbb{R}$ or $\mathbb{C}$ together with a bilinear map $[\cdot|\cdot]:L\times L\to L$ such that
\begin{enumerate}[i]
\item $[X|Y]=-[Y|X]$,
\item $[X|[Y|Z]]+[Y[Z|X]]+[Z|[X|Y]]=0$, which is called Jacobi identity.
\end{enumerate}
\end{definition}

\begin{proposition}
Let $G$ be a Lie group and $\mathfrak{g}=T_1G$. Then $\mathfrak{g}$ equipped with $[X|Y] = \ad(X)(Y)$ is a $\mathbb{R}$-Lie algebra.
\end{proposition}

\begin{proof}
Consider the commutator map $G\times G\to G,(x,y)\mapsto xyx^{-1}y^{-1}$. This is a smooth map as it is a composition of smooth maps $\mu(\mu(\cdot,\cdot),\mu(\iota(\cdot),\iota(\cdot)))$. Moreover, we can write this as
\begin{equation*}
\underline{\Ad}(x)(y)\iota(y).
\end{equation*}
Differentiate this at $y=1$ in the direction of $Y$, we get 
\begin{equation*}
d(\underline{\Ad}(x)(1)\iota(1))Y = \Ad(x)Y-Y,
\end{equation*}
since $d\iota(Y) = -Y$. Differentiate this again at $x=1$ with respect to $X$ we get $[X,Y]$.\\
\par Repeating the argument with 
\begin{equation*}
x\underline{\Ad}(y)(\iota(x)).
\end{equation*}
Differentiate this at $x=1$ with the direction to $X$ we get
\begin{equation*}
X-\underline{\Ad}(y)X=X-yXy^{-1}.
\end{equation*}
Differentiate this again at $y=1$ with the direction to $Y$, we get $-[Y|X]$. By smoothness, we get
\begin{equation*}
[X|Y]=-[Y|X].
\end{equation*}
For the second property, we consider the Lie group homomorphism,
\begin{equation*}
\Ad:G\to\GL(\mathfrak{g}).
\end{equation*}
By Proposition \ref{comp_homo}, we have
\begin{equation*}
\ad[X|Y]_G=[\ad(X)|\ad(Y)]_{\GL(\mathfrak{g})} = \ad(X)\ad(Y)-\ad(Y)\ad(X).
\end{equation*}
Therefore, by definition of $[\cdot|\cdot]$, we get
\begin{equation*}
[[X|Y]|Z]=[X|[Y|Z]]-[Y|[X|Z]]
\end{equation*}
By the first property, we get the Jacobi identity.
\end{proof}

\begin{example}
If $V$ is a finite dimensional $\mathbb{R}$-vector space then $\End(V)$ equipped with $[X|Y]=XY-YX$ is a Lie algebra. In fact, this coincides with the Lie algebra of the Lie group $\GL(V)$.
\end{example}

\begin{definition}
A homomorphism of Lie algebras is a linear map $f:L\to M$ such that for $X,Y\in L$ 
\begin{equation*}
f([X|Y]_L) = [f(X)|f(Y)]_M
\end{equation*}
\end{definition}

\begin{corollary}
If $f:G\to H$ is a homomorphism of Lie groups, then $df(1):\mathfrak{g}\to\mathfrak{h}$ is a homomorphism of Lie algebras.
\end{corollary}

\subsection{The identity component}

\begin{lemma}
Let $G$ be a topological group. If $H\subset G$ is an open subgroup, then it is also closed. Thus if $G$ is connected we have $H=G$.
\label{op_subgroup}
\end{lemma}
\begin{proof}
Let $\{1\}\cup I$ be a set of representations of equivalence classes in $G/H$. In other words we have
\begin{equation*}
G = H\cup\bigcup_{i\in I}iH. 
\end{equation*}
Since $\bigcup_{i\in I}iH$ is open, thus its complement $H$ is closed.
\end{proof}

\begin{lemma}
Let $G$ be a connected topological group and $U\subseteq G$ be a neighborhood of $1$. Then $U$ generates $G$.
\label{gen_con_top}
\end{lemma}

\begin{proof}
%Why U\cap U^{-1} is non-empty and open?
Since $U\cap U^{-1}$ is non-empty and open. We may assume with out the loss of generality that $U=U^{-1}$. Let us denote 
\begin{equation*}
U^n = \{g_1\cdots g_n\:|\: g_1,\cdots,g_n\in U\}.
\end{equation*}
And for $g_1\cdots g_n\in U^n$, we take $V\subset U$ an open subset and $g_1\in V$. %later
$Vg_2\cdots g_n$ is open in $U$. We now conclude that
\begin{equation*}
H=\bigcup_{n=1}^\infty U^n
\end{equation*}
is an open subset which is a subgroup of $G$ since it is closed under multiplication and inversion. By Lemma \ref{op_subgroup}, we conclude that $H=G$.
\end{proof}

\begin{definition}
A subgroup $H$ of a group $G$ is said to be characteristic if for any automorphism $\varphi:G\to G$, we have $\varphi(H)\subseteq H$. 
\end{definition}

\begin{proposition}
Let $G$ be a topological group and $G^0$ be the connected component of $G$ containing $1$. 
\begin{enumerate}[1)]
\item $G^0$ is a closed characteristic subgroup of $G$.
\item If $G$ is locally connected then $G^0$ is open and contained in any open subgroup of $G$.
\item The connected component of $G$ are precisely $G^0$-cosets.
\end{enumerate}
\end{proposition}

\begin{proof}
By Proposition \ref{comp_clo}, $G^0$ is a closed set. Since continuous maps preserve connectedness and $1$ is mapped to $1$ as we talk about group homomorphisms, we can conclude that $G^0$ is characteristic. Similarly, since multiplication and inversion are smooth, thus continuous, we conclude that $G^0$ is a subgroup of $G$. This proves the first statement.\\
\par If $G$ is locally connected, by Proposition \ref{comp_op_loc}, $G^0$ is open. If $H\subset G$ is any open subgroup, then $H\cap G^0$ is an open subgroup of $G^0$. By Lemma \ref{gen_con_top}, we have $H\cap G^0$ generates $G^0$. $H\cap G^0$ is a group, we conclude that it is equal to $G^0$. This shows that $G^0$ is contained in any open subgroup of $G$.\\
\par Let $C$ be a connected component and $g\in C$. Since $\mu(\cdot,g^{-1}):G\to G$ is continuous, we conclude that $\mu(C,g^{-1})$ is contained in the connected component which contains $1$. Hence $C=G^0g$.
\end{proof}

\subsection{Invariant vector fields}

\begin{definition}
Let $M$ be a manifold. A vector field $v$ on $M$ is an assignment that for each $p\in M$, we have $v(p)\in T_pM$. It is said to be smooth if locally around each point $p\in M$, it's coefficients in terms of local coordinates are smooth functions. In other words, given a chart $h:U\to V$, we can get a basis ${\frac \partial {\partial x_1}},\cdots,{\frac \partial {\partial x_n}}$ of $T_pM$ for all $p\in U$. And locally
\begin{equation*}
v(p) = \sum_{i=1}^n c_i(p)\cdot{\frac \partial {\partial x_i}}.
\end{equation*}
And each $c_i$ is smooth.
\end{definition}

\begin{definition}
Let $M$ be a manifold, $v$ be a smooth vector field. An integral curve is a pair $(I,\gamma)$ where
\begin{enumerate}[i).]
\item $I$ is an open interval,
\item $\gamma:I\to M$ is a smooth map such that $\gamma'(t) = v(\gamma(t))$.
\end{enumerate}
\end{definition}

\begin{proposition} 
Let $v$ be a smooth vector field on a manifold $M$, then we have the following statements.
\begin{enumerate}[1).]
\item Given $p\in M$, there exists a integral curve $(I,\gamma)$ such that $0\in I$ and $\gamma(0)=p$. 
\item $(I_1,\gamma_1),(I_2,\gamma_2)$ be integral curves with above properties. Then for any $t\in I_1\cap I_2$, we have $\gamma_1(t)=\gamma_2(t)$. 
\item In particular, we can splice such $\gamma_1,\gamma_2$. 
\end{enumerate}
\label{existence_integral_curve}
\end{proposition}

\begin{proof}
Follows from existence and uniqueness of solutions of ordinary differential equations in $\mathbb{R}^n$ via charts.%later.
\end{proof}

\begin{remark}
There is a maximal integral curve through $p$.
\end{remark}

\begin{definition}
Let $G$ be a Lie group and $g\in G$. We define
\begin{equation*}
L_g:G\to G, L_g(x) = gx,\quad R_g:G\to G, R_g(x)=xg,
\end{equation*}
the left and the right translations. Obviously these are diffeomorphisms as the inverses are $L_{g^{-1}}, R_{g^{-1}}$, respectively.
\end{definition}

\begin{remark}
By differentiating these we get
\begin{equation*}
dL_g(1):\mathfrak{g}\to T_gG,\quad dR_g(1):\mathfrak{g}\to T_gG.
\end{equation*}
\end{remark}

\begin{proposition}
Let $G$ be a Lie group and $g\in G$. Then $dL_g(1),dR_g(1)$ are isomorphisms between $\mathfrak{g}$ and $T_gG$. Therefore, we can naturally identify $T_gG$ by $\mathfrak{g}$. Moreover, $dL_g(1),dR_g(1)$ are not the same in general and differ by the automorphism $\Ad(g)$.
\end{proposition}

\begin{proof}
%later
\end{proof}

\begin{definition}
A vector field $v$ on the Lie group $G$ is said to be
\begin{enumerate}[1).]
\item left-invariant if $v(g) = dL_g(1)(v(1))$,
\item right-invariant if $v(g) = dR_g(1)(v(1))$.
\end{enumerate}
\end{definition}

\begin{remark}
Such vector field is automatically smooth. And the assignments 
\begin{equation*}
X^L=X\mapsto(g\to dL_g(1)(X)), \quad X^R= X\mapsto(g\to dR_g(1)(X))
\end{equation*}
identify the Lie algebra $\mathfrak{g}$ with the space of left/right-invariant vector fields on $G$.
\end{remark}

\begin{lemma}
Let $v$ be a left-invariant vector field on $G$. The maximal integral curve $\gamma$ with $\gamma(0)=1$ is defined on all of $\mathbb{R}$ and is a group homomorphism.
\label{max_int_curve}
\end{lemma}

\begin{proof}
Let $\gamma:I\to G$ be an integral curve with $v$ with $\gamma(0)=1$.\\
\par Assume $I\not=\mathbb{R}$ thus, without the loss of generality $I$ has an upper bound $t_0\in\mathbb{R}$. We will have to show that $\gamma$ is not maximal. To see this, we choose $0<\varepsilon<t_0$ and $t_0-\varepsilon<t_1<t_0, t\in I$.\\
\par Consider $\delta(t) = \gamma(t_1)\cdot\gamma(t-t_1)$. Thus $\gamma$ is a smooth curve defined an open neighborhood of $t_0$ and $\delta(t_1)=\gamma(t_1)$ and 
\begin{align*}
\delta'(t) &= d\delta(t)(1),\\
& = dL_{\gamma(t_1)}(\gamma(t-t_1))(dr(t-t_1)(1)),\\
& = dL_{\gamma(t_1)}(v(\gamma(t-t_1))),\\
& = dL_{\gamma(t_1)}(\gamma(t-t_1)(dL_{r(t-t_1)}(1)(v(1))),\\
& = dL_{\gamma(t_1)\gamma(t-t_1)}(1)(v(1)),\\
& = v(\gamma(t_1)\gamma(t-t_1)),\\
&= v(\delta(t)).
\end{align*}
Thus $\delta$ is an integral curve for $v$ defined on an open neighborhood of $t_0$ containing $t_1$ and $\delta(t_1) = \gamma(t_1)$. Therefore $\gamma$ is not maximal.\\
\par Now we are going to show that $\gamma$ is a homomorphism. For fixed $t\in \mathbb{R}$, note that the maps 
\begin{equation*}
s\mapsto\gamma(t+s),\quad s\mapsto\gamma(t)\gamma(s)
\end{equation*}
are both integral curves for $v$ with equal value at $s=0$, hence equal. 
\end{proof}

\subsection{The Exponential Maps}

\begin{proposition}
Let $X\in\mathfrak{g}$, there exists a unique group homomorphism $\gamma_X:\mathbb{R}\to G$ differentiable at $0$ and $\gamma'_X(0)=X$. It is the maximal integral curve through $1$ for both $X^L$ and $X^R$. We have $\gamma_{tX}(s)=\gamma_X(ts)$ for $t\in \mathbb{R}$. 
%Question : X^L = g\mapsto dL_g(1)X(1)
\end{proposition}

\begin{proof}
By Lemma \ref{max_int_curve}, there exist maximal integral curves for $X^L$ and $X^R$ , we denote them by $\gamma_{X^L},\gamma_{X^R}$, respectively. By Lemma \ref{existence_integral_curve}, we can assume $\gamma_{X^L}(0)=\gamma_{X^R}(0)=1$, and these are defined on the whole $\mathbb{R}$.\\
\par For uniqueness, let $\gamma:\mathbb{R}\to G$ be a group homomorphism which is differentiable at $0$ with $\gamma'(0)=X$. Then
\begin{equation}
\label{commutativity_curve}
\gamma(t)\gamma(s) = \gamma(t+s)=\gamma(s+t)=\gamma(s)\gamma(t).
\end{equation}
Fix $t$ and apply ${\frac {d} {ds}}|_{s=0}$ to see that $\gamma$ is differentiable at any $t$ in the following way
\begin{align*}
{\frac d {ds}}|_{s=0}\gamma(t)\gamma(s) &= {\frac d {ds}}|_{s=0}\gamma(t+s),\\
\Rightarrow\gamma(t)\gamma'(0) = \gamma'(t).
\end{align*}
By construction, when $\gamma=\gamma_{X^L}$ we have
\begin{align*}
\gamma'_{X^L}(t) &= dL_{\gamma_{X^L}(t)}(1)(X^L(1))\\
& = dL_{\gamma_{X^L}(t)}(1)X\\
& = L_{\gamma_{X^L}(t)}X.
\end{align*}
Similarly for $\gamma=\gamma_{X^R}$ we have
\begin{align*}
\gamma'_{X^R}(t) &= dR_{\gamma_{X^R}(t)}(1)(X^R(1))\\
& = dR_{\gamma_{X^R}(t)}(1)X\\
& = R_{\gamma_{X^R}(t)}X.
\end{align*}
%later check
By the uniqueness of solutions of ordinary differential equations, we derive that $\gamma_{X^L}=\gamma_{X^R}$.
This proves that $\gamma_{X^L},\gamma_{X^R}$ are maximal as they are defined on all $t\in\mathbb{R}$.\\
\par For the second property, we only need to check that $\gamma_{tX}(s)=\gamma_X(ts)$ coincide at $s=0$.
%later
\end{proof}

\begin{definition}
Let $G$ be a Lie group and $\mathfrak{g}=T_1G$. Then we define the exponential map
\begin{equation*}
\exp_G:\mathfrak{g}\to G, \exp_G(X) = \gamma_X(1),
\end{equation*}
where $\gamma_X$ is the integral curve of $v(g)=dL_{g}(1)X$.
\end{definition}

\begin{theorem}
$\exp_G:\mathfrak{g}\to G$ is smooth and has the following properties.
\begin{enumerate}[1).]
\item $\underline{\Ad}(x)\circ\exp_G = \exp_G\circ\Ad(x)$ for any $x\in G$.
\item $\Ad\circ\exp_G = \exp_{\GL(\mathfrak{g})}\circ\ad$.
\item $d\exp_G(0):\mathfrak{g}\to\mathfrak{g}$ is an identity $id_{\mathfrak{g}}$.
\item If $f:G\to H$ is a homomorphism of Lie groups, then $f\circ\exp_G=\exp_H\circ df(1)$. 
\item $\gamma_X(t)=\exp_G(t\cdot X)$.
\end{enumerate}
\label{exp_prop}
\end{theorem}

\begin{proof}
Look at the homework %later.
\end{proof}

\begin{proposition}
\label{exp_series}
Let $V$ be a finite dimensional $\mathbb{R}$-vector space. Then 
\begin{equation*}
\exp_{\GL(V)}:\mathfrak{gl}(V)\to\GL(V)
\end{equation*}
is given by 
\begin{equation*}
\exp_{\GL(V)}(X)=\sum_{n=0}^\infty{\frac 1 {n!}}X^n.
\end{equation*}
\end{proposition}

\begin{proof}
Homework%later
\end{proof}

\begin{corollary}
\label{exp_prop_cor}
Furthermore, we can derive the following properties of $\exp_G$,
\begin{enumerate}[1).]
\item $\Image\exp_G\subseteq G^0$.
\item $\exp_G:\mathfrak{g}\to G$ is a diffeomorphism locally around $0$.
\item If $U\subseteq\mathfrak{g}$ is a neighborhood of $0$ in $\mathfrak{g}$, then $\exp_G(U)$ generates $G^0$.
\end{enumerate}
\end{corollary}

\begin{proof}Note that $\exp_G$ is a smooth map.\\
\par By the smoothness, it is also continuous. Since $\mathfrak{g}$ is connected, it is mapped to a connected subset of $G$ which contains $1$. Thus we have the first property.\\
\par By the third property of Theorem \ref{exp_prop}, we have $d\exp_G(0)$ is invertible.\\
\par By the second property of the same theorem, $\exp_G(U)$ contains an open neighborhood of $1$, thus generates $G^0$.
\end{proof}

\begin{definition}
Let $G$ be a Lie group and $\mathfrak{g}$ be its Lie algebra. By the second statement of the corollary above, there exists a neighborhood $U$ of $0$ in $\mathfrak{g}$, such that $\exp_G|U$ is a diffeomorphism. We denote its inverse by $\log_G$.
\end{definition}

\begin{corollary}
Let $G$ be a connected Lie group and $g\in G$, we have the following
\begin{equation*}
g\in Z(G) \Leftrightarrow \Ad(g) = id_{\mathfrak{g}}.
\end{equation*}
\end{corollary}

\begin{proof}
If $g\in Z(G)$, then $\underline{\Ad}(g) = id_G$, therefore $\Ad(g)=id_{\mathfrak{g}}$. Conversely, if $\Ad(g)=id_{\mathfrak{g}}$,  by the first property of Theorem \ref{exp_prop} we have $\underline{\Ad}(g)$ is identity on the image of $\exp_G$. By the second statement of Corollary \ref{exp_prop_cor}, this image generates $G$. Since $\underline{\Ad}(g)$ is a homomorphism, it is trivial on the entire group $G$.
\end{proof}

\begin{corollary}
Let $G$ be a Lie group and $X,Y\in\mathfrak{g}$. We have
\begin{equation*}
[X|Y] = 0\Rightarrow \exp_G(X)\exp_G(Y)=\exp_G(Y)\exp_G(X).
\end{equation*}
\end{corollary}
\begin{proof}
Let $x=\exp_G(X),y=\exp_G(Y)$. By the first and second statements of Theorem \ref{exp_prop}, 
\begin{equation*}
xyx^{-1}=\exp_G(\Ad(X)Y) = \exp_G(\exp_{\GL(\mathfrak{g})}(\ad(X)(Y))).
\end{equation*}
By Proposition \ref{exp_series} and the assumption, this is equal to 
\begin{equation*}
\exp_G(Y)=y.
\end{equation*}
\end{proof}

\begin{corollary}
Let $f_1,f_2:H\to G$ be homomorphisms of Lie groups. If $H$ is connected and $df_1(1)=df_2(1)$. Then $f_1=f_2$.
\end{corollary}

\begin{proof}
Using the forth statement of Theorem \ref{exp_prop}, we have $f_1=f_2$ upon restriction to the image of $\exp_H$, and such image generates $H$. 
\end{proof}

\subsection{Differentials of $\exp_G$}

\begin{theorem}
Let $X\in\mathfrak{g}$. (Recall that we have the canonical identification $T_x\mathfrak{g}\to\mathfrak{g}$).\\
\par Consider 
\begin{equation*}
d(\exp_G)(x):\mathfrak{g}\to T_{\exp(X)}G,\quad dR_{\exp_G(x)}(1):\mathfrak{g}\to T_{\exp_G(x)}G.
\end{equation*}
Then we have the following,
\begin{equation*}
dR_{\exp_G(x)}(1)^{-1}\circ d(\exp_G)(x):\mathfrak{g}\to\mathfrak{g}, X\to\int_0^1\exp_{\GL(\mathfrak{g})}(s\cdot\ad(X))ds.
\end{equation*} 
\label{g_end_int}
\end{theorem}

\begin{proof}
%later
\end{proof}

\begin{corollary}
An element $X\in\mathfrak{g}$ is a singular point for $\exp_G$ if and only if $\ad(X)\in\mathfrak{gl}(\mathfrak{g})$ has an eigenvalue of the form $2\pi ik$ for some $k\in Z^{\times}$. 
\label{sing_lie_alg}
\end{corollary}

\begin{proof}
Since both $\mathfrak{g}$ and $G$ have the same dimension, $X$ is singular if and only if $d(\exp_G)(X)$ is not invertible. By Theorem %later%
the equation
\begin{equation}
\int_0^1\exp_{\GL(\mathfrak{g})}(s\cdot\ad(X))dx
\label{eq_int_end}
\end{equation}
is not invertible. In other words, it admits $0$ as an eigenvalue. Using the formula
\begin{equation*}
\int_0^1\exp_{\GL(\mathfrak{g})}(s\lambda)dx = 
\begin{cases}
\lambda^{-1}(e^{\lambda}-1)\quad(\lambda\not=0),\\
1\quad(\lambda=0).
\end{cases}
\end{equation*}
We see that the eigenvalues of the \eqref{eq_int_end} are given by $1$ if $0$ is an eigenvalue of $\ad(X)$ and $\lambda^{-1}(e^\lambda-1)$ if $\lambda\not=0$ is an eigenvalue of $\ad(X)$.
\end{proof}

\begin{remark}
The formula \eqref{eq_int_end} generalizes to 
\begin{equation*}
\int_0^1e^{sA}ds = A^{-1}(e^A-1)=\sum_{k=0}^\infty {\frac 1 {(k+1)!}}A^k.
\end{equation*}
for any $A\in\GL(V)$ where $V$ is a finite dimensional $\mathbb{R}$-vector space. If $A$ is not invertible, we can define $A^{-1}(e^A-1)$ by the above formula.\\
\par This is particularly useful for $A=\ad(X)$, for $X\in\mathfrak{g}=V$, which is never invertible since $\ad(X)(X)=0$. Moreover, for $A=\ad(X), A^{-1}(e^A-1)$ is invertible for $X$ in a neighborhood of $0$ by Corollary \ref{sing_lie_alg}.
\end{remark}

\subsection{The Product in Logarithmic Coordinates}

\begin{theorem}
Let $U\subset\mathfrak{g}$ be an open neighborhood of $0$. For $X,Y\in U$, consider the differential equation for $z:\mathbb{R}\to\mathfrak{g}$, such that
\begin{equation*}
z(0)=Y,\quad{\frac {dz} {dt}}(t) = (\ad z(t))^{-1}(\exp_{\GL{\mathfrak{g}}}(\ad z(t))-1))^{-1}(X).
\end{equation*}
For $U$ sufficiently small, this differential equation has (a unique) solution for all $X,Y\in U$ and all $t\in[0,1]$. Define $\mu(X,Y) = z(1)$. Then 
\begin{equation*}
\exp_G(X)\exp_G(Y) = \exp_G(\mu(X,Y)).
\end{equation*}
\label{product_exp_expression}
\end{theorem}

\begin{proof}
%later
\end{proof}

\begin{corollary}
The collection of maps $\kappa_x:U\to G$, where $U\subset\mathfrak{g}$ is an open neighborhood of $0$. 
\begin{equation*}
\kappa_x(Y)= x\cdot\exp_G(Y)
\end{equation*}
is a smooth, in fact real analytic, atlas for the manifold $G$.
\label{atlas_generated_by_exponential_map}
\end{corollary}

\begin{proof}
We know that $\exp_G$ is smooth and a locally diffeomorphism around $0$. So $\kappa_x$ is a diffeomorphism onto its image . Thus $(\kappa_x)_{x\in G}$ is a smooth atlas.\\
\par The transition maps are expressible in terms of $\mu$ by Theorem \ref{product_exp_expression}. Since $\mu$ is real analytic in $X,Y$. we see that the atlas is real analytic.
\end{proof}

\begin{definition}[Real analytic manifolds]
A manifold is said to be%later
\end{definition}

\begin{remark}
In particular, any Lie group is automatically real analytic.
\end{remark}

\begin{theorem}
Let $X,Y\in U$, then 
\begin{equation*}
\mu(X, Y) = X+Y+\sum_{k=1}^\infty {\frac {(1)^k)} {k+1}}\sum_{\substack{l_1,\cdots,l_k\geq0,\\m_1, \cdots,m_k\geq0,\\l_i+m_i>0}}
{\frac 1 {\sum_{i=1}^k l_i +1}}\prod_{i=1}^k{\frac {\ad(X)^{l_i}} {l_i!}}{\frac {\ad(X)^{m_i}} {m_i!}}
\end{equation*}
\end{theorem}

\begin{corollary}
\begin{equation*}
\mu(X,Y)=X+Y+{\frac 1 2}[X,Y]+O(|(X,Y)|^3).
\end{equation*}
\end{corollary}

\subsection{Lie Subgroups}

\begin{definition}
Let $G$ be a Lie group. A Lie subgroup $H$ of $G$ is a immersive submanifold that is also a subgroup.
\end{definition}

\begin{definition}
Let $\mathfrak{g}$ be a Lie algebra. A subspace $\mathfrak{h}$ of it is called a Lie subalgebra if it is closed under the Lie bracket operation $[\cdot|\cdot]$.
\end{definition}

\begin{definition}
Let $G$ be a Lie group, then we denote 
\begin{equation*}
\Lie(G)=T_1G.
\end{equation*}
\end{definition}

\begin{remark}
A tautological inclusion $i_H:H\to G$ is an injective immersion.%later
\end{remark}

\begin{theorem}
Let $G$ be a connected Lie group. Then there is a bijection between
\begin{equation*}
\{H\subseteq G\:|\: \text{connected Lie subgroups.}\}\leftrightarrow\{\text{Lie subalgebras of $\Lie(G)$.}\},
\end{equation*}
%which is induced by the tautological injection $i_H:H\to G$.
And the bijection is given by $\Lie(\cdot)$.
\end{theorem}

\begin{proof}
Let $H$ be a subgroup of $G$ and $\mathfrak{h}=T_1H$.\\
\par We first prove the injectivity of $\Lie(\cdot)$.\\
\par For the surjectivity, let us take $H\subseteq G$ to be a subgroup generated by the image of $\exp_G(\mathfrak{h})$ for a Lie subalgebra $\mathfrak{h}$. By Corollary \ref{atlas_generated_by_exponential_map}, we have
\begin{equation*}
(\kappa_x^{-1})_{x\in G},\quad \kappa_x(Y)=x\exp_G(Y)
\end{equation*}
is an atlas for $G$. We will show that
\begin{equation*}
(\kappa_x^{-1})_{x\in H}
\end{equation*}
is an atlas for $H$.\\
\par First, we claim that if 
\begin{equation*}
\kappa_x(U\cap\mathfrak{h})\cap\kappa_y(U\cap\mathfrak{h})\not=\emptyset,
\end{equation*}
then there exist neighborhood $V_1,V_2$ of $0$ in $\mathfrak{g}$ such that
\begin{equation*}
\kappa_y^{-1}\circ\kappa_x:V_1\cap\mathfrak{h}\to V_2\cap\mathfrak{h}
\end{equation*}
is a diffeomorphism.\\
\par Since $\kappa_x$ is an atlas for $G$, there exist some open neighborhoods $V_1,V_2$ of $0$ such that
\begin{equation*}
\kappa_y^{-1}\circ\kappa_x:V_1\to V_2
\end{equation*}
is a diffeomorphism. The above composition is given by
\begin{equation*}
\kappa_y^{-1}\circ\kappa_x(Y) = \log_G(y^{-1}x\exp_G(Y)).
\end{equation*}
Let $z=y^{-1}x$ then since $x,y\in H$, $z\in H$. Let $X\in\mathfrak{h}$ be such that
\begin{equation*}
z = \exp(X). %later why such X exists.
\end{equation*}
Thus by using Theorem \ref{product_exp_expression}, we obtain
\begin{equation*}
y^{-1}x\exp_G(Y) = z\exp_G(Y)=\exp_G(X)\exp_G(Y)=\exp_G(\mu(X,Y)).
\end{equation*}
Thus if $X,Y\in \mathfrak{h}$, then $\mu(X,Y)\in\mathfrak{h}$\\%Check this assertion may be using the uniqueness of the solution of diffeq with an initial value ?
%Revise the argument below here
\par For each $x\in H$, through $\kappa_x:\mathfrak{h}\cap U\to H$, we get a basis of a neighborhood of $0$ in $\mathfrak{h}$. This topologizes $H$ and 
\begin{equation*}
(\kappa_x^{-1})_{x\in G},\quad \kappa_x(Y)=x\exp_G(Y)
\end{equation*}
becomes an atlas. Therefore, the tautological inclusion $\iota:H\to G$ is now an immersion.
\end{proof}

\begin{remark}
This map is a bijection from the set of connected Lie subalgebras of $G$ to the set of Lie subalgebras of $\Lie(G)$. 
\end{remark}

\begin{definition}
A subset $\mathfrak{h}$ of a Lie algebra $\mathfrak{g}$ is called an ideal if for any $X\in \mathfrak{h}$ and $Y\in\mathfrak{g}$, we have
\begin{equation*}
[X|Y]\in\mathfrak{h}.
\end{equation*}
\end{definition}

\begin{lemma}
%Asked Andrea
\end{lemma}

\begin{lemma}
Given a Lie group $G$ and its Lie algebra $\mathfrak{g}$. $H$ is a normal connected Lie subgroup of $G$ if and only if $\Lie(H)$ is an ideal. 
\end{lemma}

\begin{proof}
%Homework
\end{proof}

\begin{lemma}
Let $G$ be a topological group and $H\subset G$ be a locally closed subgroup. Then $H$ is a closed as a subset of $G$.
\label{locally_closed_subgroup}
\end{lemma}

\begin{proof}
%Homework.
\end{proof}

\begin{lemma}
Let $H\subset G$ be a Lie subgroup of a Lie group $G$. Then $H$ is embedded submanifold if and only if $H$ is closed.
\end{lemma}
\begin{proof}
Since $H$ is an embedded submanifold, it is locally closed. %Check this reasoning.
By Lemma \ref{locally_closed_subgroup}, $H$ is closed. Conversely, if $H$ is closed then a tautological injection $\iota:H\to G$ is a closed injective immersion. Hence $H$ is an embedded submanifold. %Check this reasoning as wel.
\end{proof}

\begin{theorem}
Let $G$ be a Lie group and $H\subseteq G$ be a subgroup which is closed as a set. Then $H$ is a closed Lie subgroup.
\label{closed_subgroup}
\end{theorem}

\begin{proof}
Without the loss of generality, we may assume that $H$ is a connected set.\\
\par Let us define a subset of the Lie algebra $\mathfrak{g}$ as
\begin{equation*}
\mathfrak{h} = \{X\in\mathfrak{g}\:|\:\forall t\in\mathbb{R}, \exp_G(tX)\in H\}.
\end{equation*}
We will first show that $\mathfrak{h}$ is a subspace of $\mathfrak{g}$. The scalar multiplication is obvious from the definition. We need to show that this is closed under addition. \\
\par Fix $t\in\mathbb{R}$, for a large enough $n$, there is a neighborhood $V$ of $0$ in $\mathfrak{g}$ such that
\begin{equation*}
n^{-1}tX,n^{-1}tY\in V.
\end{equation*}
Since $V$ is sufficiently small, $\exp_G|V$ is a diffeomorphism and by Theorem \ref{product_exp_expression}, we have for any $H,K\in V$,
\begin{equation*}
\exp_G(H)\exp_G(K)=\exp_G(\mu(H,K)).
\end{equation*}
Differentiate the above equation at $(0,0)$, we get
\begin{equation*}
d\mu(H,K) = H+K.
\end{equation*}
Therefore we have
\begin{equation*}
n\cdot d\mu(n^{-1}tX,n^{-1}tY) = t(X+Y).
\end{equation*}
Proof of the closedness\\%later
\par Let $U\subseteq\mathfrak{h}$ be a neighborhood of $0$. Then $\exp_G(U)$ is a neighborhood of $1$ in $H$ by the construction.\\
\par Suppose there is a sequence $(h_n)_{n\in\mathbb{N}}\subset H\backslash\exp_G(U)$ which converges to $1$. Let $\mathfrak{k}$ be a complement of a vector space $\mathfrak{h}$, that is 
\begin{equation*}
\mathfrak{g} = \mathfrak{h}\oplus\mathfrak{k}.
\end{equation*}
Consider the map
\begin{equation*}
\Phi:\mathfrak{h}\oplus\mathfrak{k}\to G, \Phi(H,K) = \mu(\exp_G(H),\exp_G(K)).
\end{equation*}
Differentiating this at $(0,0)$ we get
\begin{equation*}
d\Phi(H,K) = H+K,
\end{equation*}
since $d\exp_G$ is identity at $0$. We conclude that $\Phi$ is a local diffeomorphism. Thus we can find sequences $(H_n)_{n\in\mathbb{N}}\subset\mathfrak{h} ,(K_n)_{n\in\mathbb{N}}\subset\mathfrak{k}$ such that for big enough $N$ and $n\geq N$,
\begin{equation*}
\exp_G(H_n)\exp_G(K_n) = h_n.
\end{equation*}
For such $(K_n)_{n\in\mathbb{N}}$ we have $K_n\to 0$.\\%later
Now define a sequence $K'_n = {\frac {K_n} {|K_n|}}$ in the Euclidean norm. This belongs to the unit sphere in $\mathfrak{k}$ which is a compact set. Thus, we find a subsequence $(K_{n_m})_{m\in\mathbb{N}}$ which is converging to some $K$.
\par Let $t>0$ be fixed. Since $|K'_n|\o 0$, we can find $(k_n)_{n\in\mathbb{N}}\subseteq\mathbb{N}$ such that
\begin{equation*}
k_n\leq {\frac t {|K'_n|}}\leq k_{n+1}.
\end{equation*}
Therefore, as $n\to\infty$, 
\begin{equation*}
k_n|K'_n|\to t.
\end{equation*}
Passing these to $\exp_G$ we get
\begin{equation*}
\exp_G(tK) = \lim_{m\to\infty}\exp_G(k_{n_m}K_{n_m}) = \lim_{m\to\infty}\exp_G(K_{n_m})^{k_{n_m}}.
\end{equation*}
Since $\exp_G(Y_n)=(\exp_G(H_n))^{-1}h_n$, it belongs to $H$, so does $\exp_G(tK)$.
%later
\end{proof}
\begin{corollary}
Let $f:G\to H$ be a homomorphism of groups between two Lie groups which is continuous. Then $f$ is a homomorphism of Lie groups.
\end{corollary}

\begin{proof}
%later
\end{proof}

\begin{corollary}
Let $f:G\to H$ be a homomorphism of Lie groups. Then $\Ker f$ is a closed Lie subgroup with its Lie algebra $\Ker(df(1))$.
\label{kernel_lie_subgroup}
\end{corollary}

\begin{proof}
Since $H$ is Hausdorff by definition, $\{1\}$ is closed in $H$. $f$ is continuous implies that $K=\Ker f$ is closed in $G$ as it is an inverse image of a closed set. By Theorem \ref{closed_subgroup}, it is a Lie subgroup. By the forth statement of Theorem \ref{exp_prop}, we have
\begin{equation*}
f\circ\exp_G = \exp_H\circ df(1).
\end{equation*}
Therefore we have an inclusion $\Ker(df(1))\subseteq\Lie(K)$. Conversely, let $X\in\Lie(K)$ and define a map
\begin{equation*}
t\mapsto f(\exp_G(tX))=1.
\end{equation*}
By differentiating with respect to $t$ at $0$ we see
\begin{equation*}
df(1)(X) = 0.
\end{equation*}
Therefore $X\in K$.
\end{proof}

\begin{corollary}
Let $f:G\to H$ be a homomorphism of Lie groups. If $f$ is an injection then $f$ is an immersion. Furthermore, if $f$ is a bijection then $f$ is an isomorphism.
\end{corollary}

\begin{proof}
$f$ is injective if and only if $\Ker f=\{1\}$. By Corollary \ref{kernel_lie_subgroup}, we have 
\begin{equation*}
\Ker(df(1))=\{0\}.
\end{equation*}
By translation by an arbitrary $g\in G$, %check what exactly is the translation.
\begin{equation*}
\Ker(df(g))=\{0\}.
\end{equation*}
Moreover, if $f$ is a bijection, then looking locally through charts, we see %check
\begin{equation*}
\dim(G) = \dim(H).
\end{equation*}
Therefore we have
\begin{equation*}
\dim\mathfrak{g}=\dim\mathfrak{h}.
\end{equation*}
Since $df(1)$ is an injective linear map between linear spaces with the same dimension, thus a bijection. Using translations again, we see that $df(g)$ is bijective for any $g\in G$.%check
Therefore, $f$ is a bijective local diffeomorphism everywhere, we conclude $f$ is a diffeomorphism. %check.
\end{proof}

\subsection{Group Action of Lie Groups}

\begin{definition}
Let $G$ be a Lie group and $M$ be a manifold. A smooth action of $G$ on $M$ is a smooth map $l:G\times M\to M$ which is a group action.
\end{definition}

\begin{definition}
A group action $l:G\times M\to M$ of a topological group $G$ on a manifold $M$ is said to be proper if the map
\begin{equation*}
(g,x)\mapsto (gx,x)
\end{equation*}
is proper.
\end{definition}

\begin{remark}
A group action $l:G\times M\to M$, we can define functions such as
\begin{enumerate}[1).]
\item $l_x:G\to M$ for fixed $x\in X$, $l_x(g) = gx$,
\item $l_g:M\to M$ for fixed $g\in G$, $l_g(x) = gx$.
\end{enumerate}
Clearly, both of them are continuous. Furthermore, we have
\begin{enumerate}[1).]
\item $l_x$ is injective if and only if $G_x=\{1\}$. 
\item $l^g$ is always a diffeomorphism with the inverse $m_{g^{-1}}$.
\end{enumerate}
\end{remark}

\begin{lemma}
Let $l:G\times M\to M$ be a smooth, free Lie group action. Then for any $x\in M$, $dl_x(1)$ is injective.
\label{injective_derivative_action}
\end{lemma}

\begin{proof}
Given $X\in\mathfrak{g}$, we have
\begin{equation*}
l_x(\exp_G((t+h)X)=l_x(\exp_G(tX)\exp_G(hX)) = l^{\exp_G(tX)}l_x(\exp_G(hX)).
\end{equation*}
%Proving injectivity part (Assuming X is in...
Suppose $X\in\Ker(dl_x(1))$ and fix $t_0\in\mathbb{R}$. We have
\begin{equation*}
{\frac d {dt}}\bigg{|}_{t=t_0}l_x(\exp_G(tX)) = {\frac d {dh}}\bigg{|}_{h=0}l_x(\exp_G((t_0+h)X)) = dl^{\exp_G(t_0X)}(X)(dl_x(1)(X))=0.%check the last equality
\end{equation*} 
Therefore, $l_x(\exp(tX))$ is constant but $l$ is a free group action. Therefore for any $t\in\mathbb{R}$, $\exp_G(tX)=1$. We conclude that $X=0$. 
\end{proof}

\begin{theorem}
Let $m:G\times M\to M$ be a smooth, free, proper group action. We can embed the smooth manifold structure to  the equivalence classes $G/M$ by the orbits of the action with following properties.
\begin{enumerate}[i).]
\item The topology on $G/M$ is the quotient topology induced by the canonical map $\pi:M\to G/M$.
\item For an arbitrary point $p\in G/M$, there is a neighborhood $V\subseteq G/M$ and a diffeomorphism $\pi^{-1}(V)\to G\times V$, which translates the $G$ actions on $\pi^{-1}(V)$ given by the map $l$ to the $G$ action $G\times V$ by left multiplication on $G$.%check later
\end{enumerate}
\end{theorem}

\begin{proof}
Let $S$ be a complement vector space of $dl_x(1)(\mathfrak{g})$ in $T_xM$, thus we have%check where dl_x(1)(g) belongs to
\begin{equation*}
T_xM=S\oplus dl_x(1)(\mathfrak{g}).
\end{equation*}
Choose a submanifold $N\subseteq M$ such tha $x\in N$ and t $T_xN=S$. (Such $N$ is called a slice).%Check what proposition justifies this
We first show that for sufficiently small $N$, the restricted action
\begin{equation*}
\overline{l}:G\times N\to M
\end{equation*}
is a diffeomorphism onto its image. Indeed, taking the derivative of $\overline{l}$ at $(1,x)$ we derive
\begin{equation*}
d\overline{l}(1,x):\mathfrak{g}\times T_xN\to T_xM.
\end{equation*}
By the construction of $N$ and Lemma \ref{injective_derivative_action}, this is a bijection. By translation, we have $d\overline{l}(g,y)$ is bijective for any $(g,y)\in G\times N$ close enough to $(1,x)$. \\%Check if this is due to continuity and how its proved.
Take $N$ small enough so that for each $y\in N$, $d\overline{l}(1,y)$ is bijective. Using the formula
\begin{equation*}
d\overline{l}(g,y) = dl^g\circ d\overline{l}(1,y),
\end{equation*}
we see that $d\overline{l}(g,y)$ is bijective for any $(g,y)\in G\times N$.\\
Secondly, we prove that for sufficiently small enough $N$, $\overline{l}$ is injective. Suppose that there are sequences $(y_n)_{n\in\mathbb{N}}, (z_n)_{n\in\mathbb{N}}\subset N$ and $(g_n)_{n\in\mathbb{N}},(h_n)_{n\in\mathbb{N}}\subset G$ such that 
\begin{equation*}
g_iy_i = h_iz_i, (g_i,y_i)\not=(h_i,z_i).
\end{equation*}

If $g_i=h_i$ then by the definition of group actions, $y_i=z_i$, therefore, we may assume that $g_i\not=h_i$. Let $k_i = h_i^{-1}g_i\not=1$. Then the sequence
\begin{equation*}
(k_iy_i,y_i) = (z_i,y_i)\to(x,x),
\end{equation*}
is contained in a compact subset of $M\times M$.\\ %Check this reasoning.
\par By the properness of $l$, $(k_i)_{i\in\mathbb{N}}$ is contained in a compact subset of $G$. Thus it has a convergent subsequence which converges to some $k\in G$. Therefore, with that subsequence we have
\begin{equation*}
x=\lim_{i\to\infty}z_i=\lim_{j\to\infty} k_{i_j}y_{i_j} = kx.
\end{equation*}
By the freeness of $l$, we conclude that $k=1$. However, $(k_{i_j},y_{i_j}),(1,z_{i_j})\subset G\times N$ converge to $(1,x)$. This contradicts to the injectivity of $\overline{l}$ in a sufficiently small neighborhood of $(1,x)$.\\%Check this reasoning.
\par Now we are ready to close our proof. Note that the properness of the action implies that the space $G/M$ is Hausdorff as each equivalence class is closed.\\%check the reasoning
\par We introduce an atlas by 
\begin{equation*}
\overline{l}:G\times N\to\pi^{-1}(\pi(N)).
\end{equation*}
This satisfies the desired property.%Revise the proof.
\end{proof}

\begin{corollary}
Let $H\subset G$ be a closed Lie subgroup. Then the coset spaces $H\backslash G$ and $G/H$ are both smooth manifold. Furthermore, $\pi_l:G\to H\backslash G$ and $\pi_r:G\to G/H$ are principal $H$-bundles. %Check the definition of bundles.
\end{corollary}

\begin{proof}
The left action map $l:H\times G\to G$ is a multiplication and $H\times G\to G\times G$ is proper. %Check 
\end{proof}

\begin{corollary}
Let $H\subset G$ be a closed normal Lie subgroup. Then $G/H$ is a Lie group and the canonical group homomorphism $\pi:G\to G/H$ is a Lie group homomorphism.
\end{corollary}

\begin{proof}
%later
\end{proof}

\begin{corollary}
Let $f:H\to G$ be a homomorphism of Lie groups. Then there exists a unique $\overline{f}:H/\Ker f\to G$ such that 
\begin{equation*}
\overline{f}\circ\pi =f.
\end{equation*}
And such $\overline{f}$ is an injective immersion.
\end{corollary}

\begin{proof}
By elementary group theory, there exists a unique homomorphism $\overline{f}$. Since $\pi:H\to H/\Ker f$ is a fiber bundle, %check
and $f$ is smooth, $\overline{f}$ is also smooth, we conclude that $\overline{f}$ is an injective homomorphism of Lie groups. 
\end{proof}

\begin{corollary}
Let $f:H\to G$ be a homomorphism of Lie groups. If $f$ is surjective, it is a principal fiber bundle with group $K=\Ker f$. 
\end{corollary}

\begin{proof}
%later
\end{proof}

\subsection{Classification of abelian connected Lie groups.}

\begin{definition}
A Lie algebra $L$ is abelian if $L=Z(L)$, in other words, the Lie bracket is everywhere $0$. 
\end{definition}

\begin{proposition}
A connected Lie group $G$ is abelian if and only if its Lie algebra $\Lie(G)$ is abelian.
\end{proposition}

\begin{proof}
If $G$ is abelian, then the adjoint action $\Ad:G\to \GL(\mathfrak{g})$ is constant and $\mathfrak{g}\to\End(\mathfrak{g})$ is everywhere $0$. Conversely, if $\mathfrak{g}=\Lie(G)$ is abelian then for any $X,Y\in\mathfrak{g}$, we have
\begin{equation*}
\exp_G(X+Y) = \exp_G(X)\exp_G(Y).
\end{equation*}
Thus generators of the Group $G$ commutes, we conclude $G$ is abelian. 
\end{proof}

\begin{proposition}
Let $G$ be a connected abelian Lie group then $\exp_G:\mathfrak{g}\to G$ is a surjective homomorphism and its kernel is discrete.
\end{proposition}

\begin{proof}
Let $X,Y\in\mathfrak{g}$ and consider two maps
\begin{equation*}
t\mapsto\exp_G(t(X+Y)),\quad t\mapsto\exp_G(tX)\exp_G(tY).
\end{equation*}
These maps have the same derivatives at $t=0$, namely $X+Y$.\\ %Check.
\par Since $G$ is abelian, we conclude these are homomorphisms of Lie groups. Therefore, they are equal to one another. We now have that $\exp_G$ is a group homomorphism, and $\exp_G(\mathfrak{g})$ generates $G$, therefore it is surjective. \\
\par We close the proof by stating that $\Ker\exp_G$ is a closed Lie subgroup with Lie algebra, $\Ker d\exp_G(0)=\{0\}$.
%later
\end{proof}

\begin{theorem}
Let $G$ be a connected abelian then it is isomorphic to $\mathbb{R}^a\times (S^1)^b$ for some integers $a,b$.
\end{theorem}

\begin{proof}
%later
\end{proof}

\subsection{Connected Compact Lie groups}

\begin{proposition}
Let $G$ be a connected topological group and $\alpha,\beta:[0,1]\to G$ be loops base at the identity $1$. Let us define another loop by
\begin{equation*}
\gamma(s) = \alpha(s)\beta(s), \quad s\in[0,1].
\end{equation*}
Then in the fundamental group $\pi_1(G)$,  we have
\begin{equation*}
[\alpha]*[\beta] = [\gamma]=[\beta]*[\alpha].
\end{equation*}
In particular, $\pi_1(G)$ is abelian.
\end{proposition}

\begin{proof}
Let us consider the map $H:[0,1]\times[0,1]\to G, H(s,t)=\alpha(s)\beta(t)$. 
%later
\end{proof}

\begin{proposition}
Let $G$ be a connected, locally path connected (hence globally path connected) topological group. View $(G,1)$ as a pointed space and 
\begin{equation*}
p:(X,x)\to(G,1)
\end{equation*}
be a covering space. There exists a unique topological group structure on $(X,x)$ with $x$ being the identity which makes $p$ into a group homomorphism. Furthermore, such topological group and a homomorphism has the following properties.
\begin{enumerate}[1).]
\item $\Ker(p)$ is central and discrete.
\item The group of Deck transformations of $p$ is identified with $\Ker(p)$ acting by multiplication.
\end{enumerate}
\end{proposition}

\begin{corollary}
In the above setting, if $G$ i a Lie group, the above construction makes the group $(X,x)$ into a Lie group. Such structure makes $p$ into a Lie group homomorphism and $\Ker(p)$ turns out to be countable.
\end{corollary}

\begin{proposition}
Let $G$ be a connected Lie group. Then the universal cover $\tilde{G}$ of $G$ is a connected Lie group.\\
\par Furthermore, the map $[\alpha]\mapsto\alpha(1):\pi_1(G)\to\tilde{G}$ where
%later
\end{proposition}

\begin{proposition}
Let $f:H\to G$ be a homomorphism of connected Lie groups. Then the following statements are equivalent.
\begin{enumerate}
\item $f$ is a covering,
\item $d f(1)$ is an isomorphism.
\end{enumerate}
\end{proposition}

\begin{proof}
%later
\end{proof}

\begin{proposition}
Let $H$ be a connected and simply connected Lie group and $G$ be a Lie group.\\
\par For any homomorphism $\varphi:\Lie(H)\to\Lie(G)$ of Lie algebras, there exists a homomorphism $f:H\to G$ of Lie groups such that
\begin{equation*}
d f(1) = \varphi.
\end{equation*}
\end{proposition}

\begin{proof}
%later
\end{proof}

\begin{lemma}
Connected Lie groups that have the same Lie algebra have the same universal cover.
\end{lemma}

\begin{proof}
Let $G_1,G_2$ be Lie groups and $\varphi:\Lie(G_1)\to\Lie(G_2)$ is an isomorphism. %later
\end{proof}

\subsection{Compact Lie algebras}

\begin{definition}
Let $G$ be a topological group. A finite dimensional representation of $G$ over the field $\mathbb{K}=\mathbb{R},\mathbb{C}$ is a pair $(\pi,V)$ such that
\begin{enumerate}[i).]
\item $V$ is a finite dimensional vector space,
\item $\pi:G\to\GL(V)$ is a continuous group homomorphism. %Check the topology defined on GL(V)
\end{enumerate}
\end{definition}

\begin{remark}
If $G$ is a Lie group then $\pi$ is smooth and we can take $\varphi=d\pi(1):\Lie(G)\to\End(V)=\mathfrak{gl}(V)$ a homomorphism of Lie algebras.
\end{remark}

\begin{definition}
The adjoint representation of $G$ is the homomorphism 
\begin{equation*}
\Ad:G\to\GL(\mathfrak{g})
\end{equation*}
such that 
\begin{equation*}
\varphi=\ad:\mathfrak{g}\to\mathfrak{gl}(\mathfrak{g})
\end{equation*}
is the adjoint representation of $\mathfrak{g}$ on itself.%later
\end{definition}

\begin{remark}
G is a connected Lie group then
\begin{equation*}
Z(G)=\Im(\Ad).
\end{equation*}
Thus it is a closed subgroup. We also have
\begin{equation*}
\ad(\Lie(Z(G)) = \Image(\ad).
\end{equation*}
\end{remark}

\begin{definition}
Let $L$ be a Lie algebra over a field $k$.\\
\par A finite dimensional representation of $L$ is a pair $(\varphi, V)$ consisting of 
\begin{enumerate}[i).]
\item $V$ is a finite dimensional vector space,
\item $\varphi:L\to\mathfrak{gl}(V)$ is a homomorphism of Lie algebras.
\end{enumerate}
The adjoint representation is the homomorphism 
\begin{equation*}
\ad:L\to\mathfrak{gl}(L), X\to[X,\cdot]
\end{equation*}
and $Z(L)=\Im(\ad)$.%Check if this is a proposition
\end{definition}

\begin{definition}
Let $G$ be a connected Lie group and $k$ be a field. For the Lie algebra $\mathfrak{g}=\Lie(G)$ and a finite dimensional representation $(\pi,V)$ of $\mathfrak{g}$, a bilinear form $b$% later
is said to be invariant with respect to $\pi$ if for any $v,w\in $%later
and for any $g\in G$ we have
\begin{equation*}
b(\pi(g)v,\pi(g)w) = b(v,w).
\end{equation*}
\end{definition}

\begin{definition}
Let $L$ be a Lie algebra over a filed $k$ and $(\varphi, V)$ be a finite dimensional representation of $L$.\\
\par A bilinear form $b:V\times V\to k$ is said to be invariant with respect to the representation if for all $x\in L$ and $v,w\in V$ we have
\begin{equation*}
b(\varphi(x)v,w)=-b(v,\varphi(x)w).
\end{equation*}
In particular, it is invariant if it is invariant with respect to $\ad$.
\end{definition}

\begin{remark}
$b:L\times L\to k$ is invariant if and only if 
\begin{equation*}
b([X,Y],Z) = b(X,[Y,Z]).
\end{equation*}
\end{remark}

\begin{lemma}
Let $G$ be a connected Lie group and $L=\Lie(G)$ be a Lie algebra. Let us consider the representation $(\pi,V)$ of $G$ over $\mathbb{R}$ and a invariant bilinear form $b:V\times V\to \mathbb{R}$ with respect to $\pi$. The $b$ is invariant with respect ot $\varphi = d\pi(1)$.
\end{lemma}

\begin{proof}
%later
\end{proof}

\begin{lemma}
Let $V$ be a finite dimensional $k$-vector space. We have that the bilinear form
\begin{equation*}
\tr:\mathfrak{gl}(V)\times\mathfrak{gl}(V)\to k, \quad (X,Y)\mapsto \tr(XY)
\end{equation*}
is a symmetric and invariant bilinear form. %later with respect to which
\end{lemma}

\begin{proof}
%later
\end{proof}

\begin{remark}
For map $\varphi:L\to \mathfrak{gl}(V)$ we can pull-back the trace form to get an invariant symmetric bilienar form on $L$. %later.
\end{remark}

\begin{definition}
Let $L$ be a Lie algebra and $k$ be a field. The killing form $\kappa:L\times L\to k$ is the pull-back of the trace form on the adjoint representation. In other words, for any $X,Y\in L$, 
\begin{equation*}
\kappa(X,Y) = \tr(\ad(X)\circ\ad(Y)).
\end{equation*}
\end{definition}

\begin{corollary}
The killing form $\kappa:L\times L\to k$ is asymmetric bilinear form on $L$. %later
\end{corollary}

\begin{proof}
%later
\end{proof}

\subsection{Ideals of Lie Algebras and (Semi)Simplicity}

\begin{definition}
Let $L$ be a Lie algebra. A vector subspace $I$ of $L$ is called an ideal if we have
\begin{equation*}
[I,L]\subseteq I.
\end{equation*}
\end{definition}

\begin{remark}
An ideal of a Lie algebra is a subalgebra but the converse, a subalgebra needs not be an ideal. %later
\end{remark}

\begin{proposition}
Given a Lie algebra $L$ and a representation $(\varphi, V)$ of it , the following are ideals of $L$
\begin{enumerate}
\item The center $Z(L)$.
\item $[L, L] = \Span([X,Y]\:|\: X,Y\in L)$.
\item $\Ker\varphi$.
\item The left and right kernels of an invariant bilinear form $b:L\times L\to k$.
\end{enumerate}
\end{proposition}
\begin{proof}
%later
\end{proof}

\begin{lemma}
Let $L$ be a Lie algebra and $I$ be its ideal. Then we can imbed the structure of Lie algebra on $L/I$ with the bracket operator inherited from $L$.
\end{lemma}

\begin{proof}
%later
\end{proof}

\begin{lemma}
Let $L$ be a Lie algebra and $I$ be its ideal.\\
\par Then the killing form on $L/I$ is the restriction of the killing form of $L$. In particular, when $I=Z(L)$, the projection identifies the killing form.
\end{lemma}

\begin{proof}
%later
\end{proof}

\begin{definition}
A Lie algebra $L$ is said to be simple if $L$ is not abelian and has non-trivial proper ideals. %later
\end{definition}

\begin{definition}
A Lie algebra $L$ is said to be semi-simple if it is a direct sum of simple ideals.
\end{definition}

\begin{definition}
A Lie algebra $L$ is said to be reductive if $L=Z(L)\oplus[L,L]$ and $[L,L]$ is semi-simple.
\end{definition}

\begin{lemma}
Let $L$ be a semi-simple Lie algebra. Then $L$ is the direct sum of all of its simple ideals.
\end{lemma}

\begin{proof}
%later
\end{proof}

\begin{corollary}
If $L$ is a semi-simple Lie algebra then we have the following statements.
\begin{enumerate}
\item Any ideal on quotient is semi-simple,
\item $[L,L]=L$,
\item $Z(L)=0$.
\end{enumerate}
\end{corollary}

\begin{proof}
%later
\end{proof}


\begin{proposition}
If the killing form of a Lie algebra is definite as a bilinear form, then the Lie algebra is semi-simple.
\end{proposition}

\begin{proof}
%later
\end{proof}

\begin{proposition}
If the killing form is definite then the Lie algebra is reductive.
\end{proposition}

\begin{definition}
Let $k$ be a field and $L$ be a Lie algebra over $k$.\\
\par A derivative $d:L\to L$  on $L$ is a linear map such that for any $X,Y\in L$ we have
\begin{equation*}
d([X,Y]) = [d(X),Y] +[X,d(Y)].
\end{equation*}
We further define the set of all derivatives on $L$ to be $\Der(L)$.
\end{definition}

\begin{lemma}
Such $\Der(L)$ is a subalgebra of the space $\mathfrak{gl}$ and contains the image of the adjoint representation.
\end{lemma}

\begin{proof}
%later
\end{proof}

\begin{lemma}
For any derivation $d$ on $L$, we have
\begin{equation*}
d(Z(L)) \subseteq Z(L).
\end{equation*}
\end{lemma}

\begin{proof}
%later
\end{proof}

\begin{definition}
Let $L$ be a Lie algebra over $\mathbb{R}$. Then the group of automorphisms of $L$ as a $\mathbb{R}$-vector space is denoted by $\GL(L)$.
\end{definition}

\begin{lemma}
Let $M\subseteq L$ be a subalgebra of a Lie algebra. Then for any automorphisms $\varphi:M\to M$, there is an automorphism $\overline{\varphi}:L\to L$ which is an extension of $\varphi$.
\end{lemma}

\begin{proof}
%later
\end{proof}

\begin{lemma}
Let $L$ be a Lie algebra and $\delta\in\Der(L)$. Then for the exponential map
\begin{equation*}
\exp:\mathfrak{gl}(L)\to\GL(L),
\end{equation*}
we have
\begin{equation*}
\exp(\delta)\in \GL(L).
\end{equation*}
\end{lemma}

\begin{proof}
%later
\end{proof}

\begin{lemma}
Let $M$ be a subalgebra of a Lie algebra $L$ and $\varphi\in \Aut(M)$. THen we hav
\begin{equation*}
d\varphi(1)\in\Der(L).
\end{equation*}
\end{lemma}
\begin{proof}
%later
\end{proof}

\begin{lemma}
Let $L$ be a Lie algebra. The natural action of $\Ad(L)$ on $Z(L)$ is trivial.
\end{lemma}

\begin{proof}
Let $X\in\Ad(L)$ and $Z\in Z(L)$. The we have
\begin{equation*}
\ad(X)(Z) = 0.
\end{equation*}
This shows that 
\begin{equation*}
\exp(\ad(X))\cdot Z = Z.
\end{equation*}
%later.
\end{proof}

\begin{lemma}
Let $L$ be a Lie algebra and $M$ be a subalgebra of $L$. Then $\Aut(M)$ is a closed subgroup of $\GL(L)$ with the following property.
\begin{equation*}
\Aut(M)\subseteq O(L)\cap \Der(L).
\end{equation*}
\end{lemma}

\begin{proposition}
Let $L$ be a Lie algebra over $\mathbb{R}$. Then $\GL(L)$ is a Lie group of automorphisms of $L$ as a $\mathbb{R}$-vector space. Let $L'$ be a subalgebra of $L$ and denote $\Aut(L')$ to be the automorphisms of $L'$.\\
\par $\Aut(L')$ is a closed subgroup of $\GL(L)$ and it lies in $O(k)$ with its Lie algebra $\Der(L)$.
\end{proposition}

\begin{proof}
%later
\end{proof}

\begin{lemma}
Let $L$ be a Lie algebra. There is a natural homomorphism between $\Aut(L)$ and $\Aut(L/Z(L))$. Furthermore, if $L/Z(L)$ is semi-simple, the homomorphism is an isomorphism.
\end{lemma}

\begin{proof}
%later.
\end{proof}

\subsection{Compact Lie algebra}

\begin{definition}
A Lie algebra $L$ over $\mathbb{R}$ is said to be compact if its killing form over $L/Z(L)$ is negative definite.
\end{definition}

\begin{definition}
Let $L$ be a Lie algebra. The radical of $L$ is such that
\begin{equation*}
\rad(L) = \{X\in L\:|\: [X,\cdot]=0\}.
\end{equation*}
\end{definition}

\begin{lemma}
Let $L$ be a compact Lie algebra then
\begin{equation*}
Z(L) = \rad(L).
\end{equation*}
\end{lemma}

\begin{lemma}
Let $L$ be a compact Lie algebra. Then the following are equivalent.
\begin{enumerate}
\item $Z(L)$,
\item the killing form $\kappa$ is non-degenerate (more negative definite),
\item $L$ is semi-simple.
\end{enumerate}
\end{lemma}

\begin{proof}
%later
\end{proof}

\begin{lemma}
Let $L$ be a Lie algebra. For $X\in L$ and $\delta\in \Der(L)$, we have
\begin{equation*}
[X,\delta]_{\Der(L)} = -\delta(X).
\end{equation*}
\end{lemma}

\begin{proof}
%later
\end{proof}

\begin{proposition}
Let $L$ be a compact and semi-simple then we have the following statement.
\begin{enumerate}
\item $\ad:L\to\Der(L)$ is an isomorphism,
\item $\Ad(L)=\Aut(L)^O$, in particular, it is closed in $\Aut(L)$,
\item the adjoint action of $\Aut(L)$ on its Lie algebra $\Der(L)$ is translated under $1$, which is the tautological action of $\Aut(L)$ on $L$,
\item $\Ad(L)$ has a trivial center.
\end{enumerate}
\end{proposition}

\begin{proof}
%later
\end{proof}

\begin{proposition}
The following statements are equivalent for a Lie algebra $L$.
\begin{enumerate}
\item $\Ad(L)$ is a compact Lie group.
\item For $X\in L$, the curve $\gamma:\mathbb{R}\to\mathfrak{gl}(L), \gamma(t) = \exp(\ad(tX))$ is bounded.
\item For $X\in L$, $\ad(X)\in\mathfrak{gl}(L)$ is diagonalizable with purely imaginary eigenvalues.
\item $L$ is a compact Lie algebra.
\end{enumerate}
\end{proposition}

\begin{proof}
%later
\end{proof}

\begin{proposition}
Let $G$ be a Lie group and $\mathfrak{g}$ be its Lie algebra.\\
\par If $G$ is compact, then $\mathfrak{g}$ is compact.\\
\par Also if $\mathfrak{g}$ is compact and semi-simple, then $G$ is compact.
\end{proposition}

\begin{proof}
%later
\end{proof}

\subsection{Integration and Complete Reducibility}

\begin{definition}
A finite dimensional representation of a group of Lie algebra is called irreducible if it has no non-zero proper subspace. It is said to be completely irreducible if it is the direct sum of irreducibles, and indecomposable if it is not the direct sum of two proper non-zero invariant subgroups. %refine the def later
\end{definition}

\begin{example}
Let us define a map $\varphi:\mathbb{R}\to\End(\mathbb{R}^n)$,
\begin{equation*}
\varphi(x) = 
\begin{pmatrix}
1&x\\
0&1
\end{pmatrix}
.
\end{equation*}
%later
\end{example}

\begin{theorem}
Let $G$ be a compact topological group. Then there exists a unique linear functional 
\begin{equation*}
\mathcal{C}(G,\mathbb{C})\to\mathbb{C},\quad f\mapsto \int_G f(g)dg
\end{equation*}
such that
\begin{enumerate}
\item for any $g\in G$, $f(g)>0$ then $\int_G f dg >0$,
\item for each $g\in G$, $\int_G f(gh)dh = \int_G f(g)dg = \int_G f(hg)dh$,
\item $\int_G 1_G dg = 1$.
\end{enumerate}
\end{theorem}

\begin{proof}
%later
\end{proof}

\begin{proposition}
Let $G$ be a compact topological group and $H$ be its closed subgroup. Then we have
\begin{equation*}
\int_G f(g)dg = \int_{G/H}\int_H f(gh)dhd(g+H).
\end{equation*}
%later
\end{proposition}

\begin{proof}
%later
\end{proof}

\begin{lemma}
Let $G$ be a compact topological group and $(\pi,V)$ be a finite dimensional representation of $G$.\\
\par Then there exists a $G$-invariant scalar product on $V$.
\end{lemma}

\begin{proof}
%later
\end{proof}

\begin{corollary}
Let $G$ be a compact Lie group. There exists a $G$-invariant scalar product on $\Lie(G)$. %later
\end{corollary}

\begin{proof}
%later
\end{proof}

\begin{proposition}
Let $G$ be a compact topological group and $(\pi,V)$ be a finite dimensional representation of it. Then $G$ is completely irreducible.
\end{proposition}
\begin{proof}
%later
\end{proof}

\subsection{Special Examples}

\begin{definition}
We have the following groups
\begin{equation*}
\SL_2(\mathbb{R}) = \{A\in\GL_2(\mathbb{R})\:|\: \det(A) = 1\}, \quad \mathfrak{sl}_2(\mathbb{R})=\{A\in R^{2\times 2}\:|\: \tr A = 0\}.
\end{equation*}
\end{definition}

\begin{lemma}
Let 
\begin{equation*}
H = \begin{pmatrix} 1&0\\0&-1\end{pmatrix},\quad E=\begin{pmatrix} 0&01\\0&0\end{pmatrix},\quad F=\begin{pmatrix} 0&0\\1&-0\end{pmatrix}.
\end{equation*}
Then $\{H,E,F\}$ forms a basis in $\mathfrak{sl}_2(\mathbb{R})$ with following property,
\begin{equation*}
\kappa(E,F) = 4,\quad\kappa(E,E)=\kappa(F,F)=0,\quad\kappa(H,H) = 8.
\end{equation*}
Furthermore $\mathfrak{sl}_2(\mathbb{R})$ is not compact if and only if $\kappa $ is not negative definite.
\end{lemma}

\begin{proof}
%later
\end{proof}

\begin{definition}
\begin{equation*}
\SU_2(\mathbb{R})=\{A\in\SL_2(\mathbb{C})\:|\: v,w\in\mathbb{C}^2, \langle Av,Aw\rangle = \langle v,w\rangle\}
\end{equation*}
where
\begin{equation*}
v = (v_1,v_2), w = (w_1,w_2), \langle v,w\rangle = v_1\overline{w_1}+v_2\overline{w_2}.
\end{equation*}
\end{definition}

\begin{lemma}
\begin{equation*}
\SU_2(\mathbb{R}) = \left\{\begin{pmatrix}a&-b\\ b&\overline{a}\end{pmatrix}\:\bigg{|}\: a,b\in\mathbb{C}, |a|^2+|b|^2 = 1\right\},
\end{equation*}
which is a connected and simply connected Lie group. And by immediate computation we see that 
\begin{equation*}
\SU_2(\mathbb{R})\cong S^3\subseteq\mathbb{R}^4.
\end{equation*}
\end{lemma}

\begin{proof}
%later
\end{proof}

\begin{lemma}
\begin{equation*}
\mathfrak{su}_2(\mathbb{R})=\{A\in\mathbb{C}^{2\times 2}\:|\: A^* = -A,\tr(A) = 0\} = \left\{\begin{pmatrix}ia&b\\-\overline{b}&-ia\end{pmatrix}\:\bigg{|}\: a\in\mathbb{R},b\in\mathbb{C}\right\}.
\end{equation*}
which is a vector space of dimension $3$. We can take a basis
\begin{equation*}
I = \begin{pmatrix}i&0\\0&-i\end{pmatrix},J = \begin{pmatrix}0&-1\\1&0\end{pmatrix},I = \begin{pmatrix}0&-i\\-i&0\end{pmatrix}.
\end{equation*}
By calculating we derive
\begin{equation*}
[I,J]=2K,\quad [J,K]-2J],\quad [I,K]=2I.
\end{equation*}
And we also have for any $X\in\mathfrak{su}_2(\mathbb{R})$
\begin{equation*}
\kappa(X,X) = -2(a^2+b\overline{b})<0,
\end{equation*}
thus it is a negative definite. We conclude $\mathfrak{su}_2(\mathbb{R})$ is compact.
\end{lemma}

\begin{proof}
%later
\end{proof}

\begin{remark}
Let $V$ be a finite dimensional $\mathbb{R}$-vector space. We denote
\begin{equation*}
V_{\mathbb{C}} = V\tens{\mathbb{R}}\mathbb{C},
\end{equation*}
which is a finite dimensional $\mathbb{C}$-vector space. In particular 
\begin{equation*}
\dim_\mathbb{C}(V_\mathbb{C}) = \dim_\mathbb{R}(V).
\end{equation*}
Let $W\subset V$ be a subspace. Then the map
\begin{equation*}
W\mapsto W\tens{\mathbb{R}}\mathbb{C} = W_{\mathbb{C}}
\end{equation*}
defines a bijection between the subspaces of $V$ and the subspaces of $V_\mathbb{C}$ which is $\sigma$-invariant where
$\sigma$ is the complex conjugation.
\end{remark}

\begin{definition}
Let $L$ be a Lie algebra and $e,f,h\in L$ be elements such that 
\begin{equation*}
[h,e]=2e,\quad[h,f]=-2f,\quad[e,f]=h.
\end{equation*}
We call such $(e,f,h)$ a $\mathfrak{sl}_2$-triplet.
\end{definition}

\begin{definition}
Let $L$ be a Lie algebra and $i,j,k\in L$ be elements such that 
\begin{equation*}
[i,j]=2k,\quad[i,h]=-2j,\quad[j,k]=2i.
\end{equation*}
We call such $(i,j,k)$ a $\SU_2$-triplet.
\end{definition}

\begin{remark}
If $(e,h,f)$ is a $\mathfrak{sl}_2$-triplet in $L$. We get a homomorphism, similarly for a $\SU_2$-triplet $(i,j,k)$.
\end{remark}

\begin{example}[Representations of $\mathfrak{sl}_2(\mathbb{C})$]
%later
\end{example}

\begin{proposition}
Every finite dimensional representation of $\mathfrak{sl}_2(\mathbb{C})$ is isomorphic to a unique %later
\end{proposition}

\begin{lemma}
Let $(\pi,V)$ be a representation of $\mathfrak{sl}_2(\mathbb{C})$ and let $\rho = d\pi(1)$. Then the following are equivalent for a subspace $W$ of $V$.
\begin{enumerate}
\item $W$ is $\rho$-invariant.
\item $W$ is $\pi$-invariant.
\end{enumerate}
\end{lemma}

\begin{theorem}
Every finite dimensional representation of $\mathfrak{sl}_2(\mathbb{C})$ id completely reducible.
\end{theorem}

\begin{definition}
Let $V$ be a finite dimensional representation of $\mathfrak{sl}_2(\mathbb{C})$. An eigenvalue of $h$ of the triplet is called a weight. The dimension of the eigenspace corresponding to the eigenvalue is called the multiplicity of the weight.
\end{definition}

\begin{lemma}
Let $(\pi,V)$ be a finite representation of $\mathfrak{sl}_2(\mathbb{C})$. Then we have the following statements
\begin{enumerate}
\item The weights of $V$ are integers.
\item The isomorphic classes of $V$ is determined by the function $m:\mathbb{Z}\to \mathbb{N}$ which sends a weight to its multiplicity.
\item $m(1)$ is equal to the number of irreducible constant of $V$ with odd weights.
\item $m(0)$ is equal to the number of irreducible constant of $V$ with even weights.
\end{enumerate}
\end{lemma}

\begin{definition}
Let $L$ be a semi-simple Lie algebra over $\mathbb{C}$. %later
\end{definition}

\begin{definition}
Let $L$ be a Lie algebra. $X\in L$ is said to be $\ad$-semi-simple if $\ad(X)\in\End(L)$ is diagonalizable. 
\end{definition}

\begin{definition}
Let $L$ bea. Lie algebra. $T\subseteq L$ is called a Cartan subalgebra if $T$ is maximal abelian subalgebra consisting of $\ad$-semi-simple elements.
\end{definition}

\begin{lemma}
Let $\mathfrak{g}$ be a real compact semi-simple Lie algebra and $L=\mathfrak{g}\tens{\mathbb{R}}\mathbb{C}$. Then the map
\begin{equation*}
\varphi:\mathfrak{g}\to L,\quad \varphi(t) = t\tens{\mathbb{R}}\mathbb{C}=T
\end{equation*}
is a bijection between
\begin{equation*}
\{\text{Maximal abelian subalgebras }\mathfrak{t}\subseteq\mathfrak{g}\}\leftrightarrow \{\text{ Complex conjugate invariant maximal subalgebra }T\subseteq L\}.
\end{equation*}
Every such $T$ consists of $\ad$-semi-simple elements.
\end{lemma}

\begin{definition}
Let $L$ be a Lie algebra. For any subspace $V$ of $L$, we define
\begin{enumerate}
\item the centralizer of $V$ is $Z_L(V)=\{X\in L\:|\: [X,V] = 0\}$,
\item the normalizer of $V$ is $N_L(V)=\{X\in L\:|\: [X,L]\subseteq V\}$
\end{enumerate}
\end{definition}

\begin{lemma}
Let $L$ be a Lie algebra and $T\subseteq L$ be a Cartan subalgebra. Then we have
\begin{equation*}
Z_L(T) = T.
\end{equation*}
\end{lemma}

\begin{definition}
Let $L$ be a Lie algebra and $T$ be a Cartan subalgebra. For any $\alpha\in T^* = \Hom_{\mathbb{C}}(T,\mathbb{C})$, we define
\begin{equation*}
L_\alpha = \{X\in L\:|\: \forall t\in T, \ad(t)(X)=d(t)\cdot X\}.
\end{equation*}
\end{definition}

\begin{theorem}
Let $L$ be a Lie algebra and $T$ be a Cartan subalgebra. Then we have
\begin{equation*}
L = \bigoplus_{\alpha\in T^*} L_\alpha.
\end{equation*}
\end{theorem}

\begin{remark}
\begin{equation*}
L_0 = Z_L(T) = T.
\end{equation*}
\end{remark}

\begin{definition}
Let $L$ be a Lie algebra and $T$ be its Cartan subalgebra. We define
\begin{equation*}
\Phi(T,L) = \{\alpha\in T^*\:|\: \alpha\not=0, L_\alpha\not=0\}.
\end{equation*}
\end{definition}

\begin{lemma}
Let $L$ be a Lie algebra and $T$ be its Cartan subalgebra. We have $\Phi(T,L)$ is finite and generates $T^*$.
\end{lemma}

\begin{remark}
By definition $\Phi(T,L)\subseteq T*$ is finite and 
\begin{equation*}
L= T\oplus\bigoplus_{\alpha\in\Phi(T,L)}L_\alpha.
\end{equation*}
\end{remark}

\begin{lemma}
Let $L$ be a Lie algebra and $T$ be its Cartan subalgebra. For $\alpha,\beta\in T^*$, we have
\begin{enumerate}
\item $[L_\alpha,L_\beta]\subseteq L_{\alpha+\beta}$,
\item if $X\in L_\alpha, \alpha\not=0$, then $\ad(X)$ is nilpotent,
\item if $\alpha=-\beta$ then $L_\alpha\perp_{\kappa} L_\beta$.
\end{enumerate}
\end{lemma}

\begin{corollary}
Let $L$ be a Lie algebra and $T$ be its Cartan subalgebra. We have
\begin{equation*}
\alpha\in\Phi(T,L)\Rightarrow -\alpha\in\Phi(T,L).
\end{equation*}
Furthermore, we have
\begin{equation*}
L = T\oplus\bigoplus_{\pm\alpha\in \Phi(T,L)/I} (L_\alpha\oplus L_{-\alpha}).
\end{equation*}
\end{corollary}

\begin{corollary}
Let $\mathfrak{g}$ be a Lie algebra and $\mathfrak{t}\subseteq \mathfrak{g}$, we set
\begin{equation*}
L = \mathfrak{g}\tens{\mathbb{R}}\mathbb{C}, \quad T = \mathfrak{t}\tens{\mathbb{R}}\mathbb{C}.
\end{equation*}
The restriction of $\kappa$ to each summand in
\begin{equation*}
L = T\oplus\bigoplus_{\pm\alpha\in \Phi(T,L)/I} (L_\alpha\oplus L_{-\alpha}).
\end{equation*}
is Lie algebra%check
\\
\par And $\kappa:L_\alpha\to\mathbb{C}$ is a perfect pairing.
\end{corollary}

\begin{lemma}
Let $L$ be a Lie algebra and $T$ be its Cartan subalgebra. For $X\in L_\alpha$ and $Y\in L_{-\alpha}$ we have
\begin{equation*}
[X,Y] = \kappa(X,Y)\cdot t_{\alpha}.
\end{equation*}
\end{lemma}

\begin{remark}
Let $\sigma$ denotes the complex conjugation. Then $\sigma$ acts on a Lie algebra $L$ over $\mathbb{C}$ which preserves a Cartan subalgebra $T$. In deed we have
\begin{equation*}
\alpha\in\Phi(T,L),\quad (\sigma\alpha)(t) = \sigma(\alpha(\sigma^{-1}(t)).
\end{equation*}
\end{remark}

\begin{lemma}
Let $L$ be a Lie algebra, and $T$ be its Cartan subalgebra, and $\sigma$ be a complex conjugate. We have
\begin{enumerate}
\item For any $\alpha\in\Phi(T,L)$, $\sigma\alpha=-\alpha$.
\item $(i\cdot t_\alpha)\alpha\in\Phi(T,L)$ generates $T^\sigma$, and $(i\cdot\alpha)_\alpha$ generates $T^{*,\sigma} = T^*$.
\item $\kappa(t_\alpha,t_\alpha)>0$.
\item $h_\alpha = {\frac {2t_\alpha} {\kappa(t_\alpha,t_\alpha}}$ is well-defined, $\kappa(h_\alpha,h_\alpha)>0$, $\alpha(h_\alpha)=2$.
\item For any $\alpha\in\Phi(T,L)$ there is some $e\in L_\alpha$ such that $[e,\sigma(e)]=-h_{\alpha}$.
\item For such $e$ and $f=-\sigma(e),h=h_\alpha$ is a $\mathfrak{sl}_2$-triplet.
\end{enumerate}
\end{lemma}

\begin{remark}
There is not $e\in L_\alpha$ such that
\begin{equation*}
[e,\sigma(e)] = h_\alpha.
\end{equation*}
\end{remark}

\begin{proposition}
Let $L$ be a Lie algebra and $T$ be its Cartan subalgebra. For $\alpha,\beta\in\Phi(T,L)$ we have
\begin{enumerate}
\item $\dim L_\alpha = 1$,
\item If $c\in\mathbb{C}$ is such that $c\alpha\in \Phi(T,L)$ then $c=\pm1$.
\item If $\alpha+\beta\in\Phi(T,L)$ then $[L_\alpha,L_\beta] = L_{\alpha+\beta}$.
\item $\beta(h_\alpha)\in\mathbb{Z}$ and $\beta-\beta(h_\alpha)\alpha\in\Phi(T,L)$.
\item Assume $\alpha\not=\pm\beta$, let $r,q\in\mathbb{N}$ be the largest such that $\beta-r\alpha,\beta+q\alpha\in\Phi(T,L)$, then $\beta+n\alpha\in\Phi(T,L)\Leftrightarrow n\in[-r,q]\cap\mathbb{Z}$ and $r-q = \beta(h_\alpha)$.
\end{enumerate}
\end{proposition}

\begin{corollary}
%Find the definition of complexification.
Let $(L,T)$ be a complexification of a compact semi-simple Lie algebra $\mathfrak{g}$ and its maximal abelian subalgebra $\mathfrak{t}$. Then $\Phi(T,L)$ is a reduced root system.
\end{corollary}

\begin{proposition}
Let $L$ be a complex semi-simple Lie algebra and $T$ be its Cartan subalgebra. If $(L,T)$ is the complexification of $(\mathfrak{g},\mathfrak{t})$, the following statements hold.
\begin{enumerate}
\item $\Phi(T,L)$ is a reduced root system.
\item For $\alpha\in\Phi(T,L)$ we have $\dim_\mathbb{C}L_\alpha=1$ and there exist $X_\alpha\in L_\alpha,X_{-\alpha}\in L_{-\alpha}$ such that $[X_\alpha,X_{-\alpha}]=h_\alpha=\alpha^\lor$.
\item If for $\alpha,\beta\in\Phi(T,L)$, $\alpha+\beta\in\Phi(T,L)$ hold, then $[L_\alpha,L_\beta] = L_{\alpha+\beta}$.
\end{enumerate}
\end{proposition}

\begin{theorem}
Let $L$ be a complex semi-simple Lie algebra and $T$ be its Cartan subalgebra. If $(L,T)$ is the complexification of $(\mathfrak{g},\mathfrak{t})$, then for any $\alpha\in\Phi(T,L)$, there is non-zero $X_\alpha\in L_\alpha$ such that 
\begin{equation*}
[X_\alpha,X_{-\alpha}] = h_\alpha.
\end{equation*}
Furthermore, for any $\alpha,\beta\in \Phi(T,L)$ whose sum $\alpha+\beta$ also belongs to $\Phi(T,L)$, we have
\begin{equation*}
[X_\alpha,X_\beta] = (r_{\alpha,\beta}+1)x_{\alpha+\beta}.
\end{equation*}
Also the above condition is equivalent to 
\begin{equation*}
{\frac {[X_\alpha+X_\beta]} {X_{\alpha+\beta}}}=-{\frac {[X_{-\alpha}+X_{-\beta}]} {X_{-\alpha-\beta}}}.
\end{equation*}
Let us pick a set of simple roots $\Delta\subset\Phi$. Then there is a collection $(X_\alpha)_{\alpha\in\Phi(T,L)}$ which is an extension of $(X_\alpha)_{\alpha\in\Delta}$. This extension is unique up to signs.\\
\par The signs $\varepsilon_{\alpha,\beta}$ satisfy
\begin{enumerate}
\item $\varepsilon_{-\alpha,-\beta} = -\varepsilon_{\alpha,\beta}=\varepsilon_{\beta,\alpha}$.
\item $\alpha,\beta,\gamma\in \Phi(T,L)$, $\alpha+\beta+\gamma=0\Rightarrow \varepsilon_{\alpha,\beta}=\varepsilon_{\beta,\gamma}=\varepsilon_{\gamma,\alpha}$.
\end{enumerate}

If we choose a set of simple roots $\Delta\subseteq\Phi(T,L)$ and a total order on $\Delta$ the signs $\varepsilon_{\alpha,\beta}$ are completely determined by $\varepsilon_{\alpha,\beta}=1$ where $\alpha,\beta\in\Phi(T,L)$, $\alpha\in\Delta$ and $\beta\in \Phi^+(T,L)$. There is not $\alpha'\in\Delta,\beta'\in\Phi^+(T,L)$ such that $\alpha+\beta=\alpha'+\beta$ and $\alpha'<\alpha$.
\end{theorem}

\end{document}