\documentclass{article}

\usepackage{amsmath}
\usepackage{amssymb}
\usepackage{amsthm}
\usepackage{enumerate}
\usepackage{bbm}
\usepackage{lipsum}
\usepackage{fancyhdr}
\usepackage{calrsfs}
\usepackage{tikz-cd} 

\newtheorem{theorem}{Theorem}[section] 
\newtheorem{proposition}{Proposition}[section] 
\newtheorem{definition}{Definition}[section] 
\newtheorem{lemma}{Lemma}[section] 
\newtheorem{notation}{Notation}[section] 
\newtheorem{remark}{Remark}[section] 
\newtheorem{corollary}{Corollary}[section] 
\newtheorem{terminology}{Terminology}[section] 
\newtheorem{example}{Example}[section] 
\numberwithin{equation}{section}

\DeclareMathOperator{\diam}{diam}
\DeclareMathOperator{\rk}{rk}
\DeclareMathOperator{\rank}{rank}
\DeclareMathOperator{\Isom}{Isom}
\DeclareMathOperator{\Hom}{Hom}
\DeclareMathOperator{\Dom}{Dom}
\DeclareMathOperator{\grad}{grad}
\DeclareMathOperator{\Span}{Span}
\DeclareMathOperator{\interior}{int}
\DeclareMathOperator{\ind}{ind}
\DeclareMathOperator{\supp}{supp}
\DeclareMathOperator{\sgn}{sgn}
\DeclareMathOperator{\ob}{ob}
\DeclareMathOperator{\Spec}{Spec}
\DeclareMathOperator{\PreSh}{PreSh}
\DeclareMathOperator{\Fun}{Fun}
\DeclareMathOperator{\Ker}{Ker}
\DeclareMathOperator{\Image}{Im}
\DeclareMathOperator{\Ad}{Ad}
\DeclareMathOperator{\ad}{ad}
\DeclareMathOperator{\End}{End}
\DeclareMathOperator{\GL}{GL}
\DeclareMathOperator{\SL}{SL}
\DeclareMathOperator{\SU}{SU}
\DeclareMathOperator{\Lie}{Lie}
\DeclareMathOperator{\tr}{tr}
\DeclareMathOperator{\Der}{Der}
\DeclareMathOperator{\Aut}{Aut}
\DeclareMathOperator{\rad}{rad}
\DeclareMathOperator{\id}{id}


\title{V4A3 Sheet 7}
\author{So Murata}
\date{}

\begin{document}
\maketitle 
\subsection*{(i)}
Given two roots $\alpha,\beta\in R$, we have
\begin{equation*}
\langle\alpha^\lor,\beta\rangle = {\frac {2(a,b)} {(a,a)}} = {\frac {2\Vert \beta\Vert} {\Vert \alpha\Vert}}\cos\theta\in\mathbb{Z}.
\end{equation*}
Similarly for $\beta,\alpha$, we derive
\begin{equation*}
\langle\alpha^\lor,\beta\rangle\langle\beta^\lor,\alpha\rangle = 4\cos^2\theta\in\mathbb{Z}.
\end{equation*}

Thus the possibilities of $\cos\theta$ are
\begin{equation*}
\pm 1, \pm{\frac 1 2}, \pm{\frac {\sqrt{3}} 2},{\frac 1 {\sqrt{2}}}
\end{equation*}

Thus angles corresponding them in $0\leq\theta\leq\pi$ are
\begin{equation*}
{\frac {1} {6}}\pi,{\frac {1} {3}}\pi,{\frac {1} {4}}\pi,{\frac {1} {2}}\pi,{\frac {2} {3}}\pi,{\frac {3} {4}}\pi,{\frac {5} {6}}\pi.
\end{equation*}

\subsection*{(ii)}

Let $\Delta=\{\alpha,\beta\}$ be its simple roots. Then the angle between $\alpha,\beta$ are only the angles listed above. If the root system $R$ is reducible then $\Span R = \Span R_1\oplus\Span R_2$. $\Span R$ has a dimension $2$, therefore $\Span R_1,\Span R_2$ both have dimension $1$ and correspond to the standard basis. This is only possible when the angle between them are ${\frac 1 2}\pi$ with $\Delta=\{2e_1,2e_2\}, R=\{\pm 2e_1,\pm2e_2\}$.

\subsection*{(iii)}

We now have to show that each irreducible root systems have angles corresponding to 

\begin{equation*}
{\frac {1} {6}}\pi,{\frac {1} {3}}\pi,{\frac {1} {4}}\pi,{\frac {2} {3}}\pi,{\frac {3} {4}}\pi,{\frac {5} {6}}\pi.
\end{equation*}

As we are talking about the reduced and irreducible root system, we can examine the Dynkin diagram to get all the possibilities.

\begin{enumerate}
\item For ${\frac 2 3}\pi$, it corresponds to $A_2$,
\item For ${\frac 5 6}\pi$, it corresponds to $G_2$,
\item For ${\frac 3 4}\pi$, it corresponds to either $B_2,C_2$ depending on the direction of the arrow.
\end{enumerate}

\end{document}