\documentclass{article}

\usepackage{amsmath}
\usepackage{amssymb}
\usepackage{amsthm}
\usepackage{enumerate}
\usepackage{bbm}
\usepackage{lipsum}
\usepackage{fancyhdr}
\usepackage{calrsfs}
\usepackage{tikz-cd} 

\newtheorem{theorem}{Theorem}[section] 
\newtheorem{proposition}{Proposition}[section] 
\newtheorem{definition}{Definition}[section] 
\newtheorem{lemma}{Lemma}[section] 
\newtheorem{notation}{Notation}[section] 
\newtheorem{remark}{Remark}[section] 
\newtheorem{corollary}{Corollary}[section] 
\newtheorem{terminology}{Terminology}[section] 
\newtheorem{example}{Example}[section] 
\numberwithin{equation}{section}

\DeclareMathOperator{\diam}{diam}
\DeclareMathOperator{\rk}{rk}
\DeclareMathOperator{\rank}{rank}
\DeclareMathOperator{\Hom}{Hom}
\DeclareMathOperator{\Dom}{Dom}
\DeclareMathOperator{\grad}{grad}
\DeclareMathOperator{\Span}{Span}
\DeclareMathOperator{\interior}{int}
\DeclareMathOperator{\ind}{ind}
\DeclareMathOperator{\supp}{supp}
\DeclareMathOperator{\sgn}{sgn}
\DeclareMathOperator{\ob}{ob}
\DeclareMathOperator{\Spec}{Spec}
\DeclareMathOperator{\PreSh}{PreSh}
\DeclareMathOperator{\Fun}{Fun}
\DeclareMathOperator{\Ker}{Ker}
\DeclareMathOperator{\Image}{Im}
\DeclareMathOperator{\Ad}{Ad}
\DeclareMathOperator{\ad}{ad}
\DeclareMathOperator{\End}{End}
\DeclareMathOperator{\GL}{GL}
\DeclareMathOperator{\SL}{SL}
\DeclareMathOperator{\Lie}{Lie}


\title{Representation Theory 1 V4A3 Sheet 3 Exercise 1}
\author{So Murata}
\date{2024/2025 Winter Semester - Uni Bonn}

\begin{document}
\maketitle

\section*{1.}

By the property of the exponential map we have
\begin{equation*}
\underline{\Ad}(g)\circ \exp_H = \exp_H\circ\Ad(g).
\end{equation*}
We have $\exp_H(\mathfrak{h})$ generates $H$. Thus $\underline{\Ad}(g)\circ\exp_H(\mathfrak{h})$ generates $gHg^{-1}$. With the equation above, we conclude the statement.

\section*{2.}

Let $X\in\mathfrak{g}$ and $Y\in\mathfrak{h}$. Then $\exp_G(X)\in G$ and $\exp_G(Y)\in H$. By using properties of $\exp$,
\begin{align*}
\underline{\Ad}(\exp_GX)\exp_GY &= \exp_G\Ad(\exp_GX)Y,\\
&=\exp_G\exp_{\GL(\mathfrak{g})}\ad(X)(Y),\\
& = \exp_G(X)\exp_G(Y)\exp_G(X)^{-1},\\
& = \exp_G(X)\exp_G(Y)\exp_G(-X).
\end{align*}

We let
\begin{equation*}
Y'=\exp_{\GL(\mathfrak{g})}\ad(X)(Y) = \sum_{n=0}^\infty {\frac 1 {n!}}\ad(X)^n Y
\end{equation*}

If $\mathfrak{h}$ is an ideal, then $Y'\in \mathfrak{h}$. Since $\exp_G(\mathfrak{g})$ and $\exp_G(\mathfrak{h})$ generate $G$ and $H$, respectively. We conclude $H$ for $\mathfrak{h}=\Lie(H)$ is normal.\\
\par On the other hand,  we have
\begin{align*}
{\frac d {dt}}|_{t=0}\exp_{\GL(\mathfrak{g})}\ad(tX)(Y) & = \exp_{\GL(\mathfrak{g})}\ad(tX)(Y)-exp_{\GL(\mathfrak{g})}\ad(0)(Y)\mod{o(t)}\\
& = \sum_{n=0}^\infty {\frac 1 {n!}}\ad(tX)^n Y-1\mod{o(t)}\\
& = \ad(X)(Y).
\end{align*}

Since $H$ is normal for any 
\begin{equation*}
\exp_G\exp_{\GL(\mathfrak{g})}\ad(tX)(Y) = \exp_G(tX)\exp_G(Y)\exp_G(-tX)\in H.
\end{equation*}

for any $t$ thus, $\ad(X)(Y)\in\mathfrak{h}$.


\section*{Exercise 3}
Since $H$ is 1-dimensional, each chart has its terminal set a 1-dimensional vector space. We know that we can have an atlas such that
\begin{equation*}
(\kappa_x(Y)=x\exp_G(Y))_{x\in H}.
\end{equation*}
where $Y\in U$ some open neighborhood of $0$. We conclude that $U$ is 1-dimensional thus generated by a single element $X$.\\
\par Now we prove that if $H$ is closed, one of three properties will hold. Since $X\in\mathfrak{gl}(V)$, we can use the explicit form of the exponential map. Suppose $X$ is diagonalizable in $\mathbb{C}$. If there is $t\not=0$ such that $\exp_G(tX)=1$ then for any imaginary eigenvalue $b_1,b_2$ we have
\begin{equation*}
tb_j = 2k_j\pi i\quad(j=1,2).
\end{equation*}

Thus ${\frac {b_1} {b_2}}$ is a rational number. If such $t$ doesn't exist then there exists an eigenvalue which is not purely imaginary. 



\end{document}