\documentclass{article}

\usepackage{amsmath}
\usepackage{amssymb}
\usepackage{amsthm}
\usepackage{enumerate}
\usepackage{bbm}
\usepackage{lipsum}
\usepackage{fancyhdr}
\usepackage{calrsfs}
\usepackage{tikz-cd} 

\newtheorem{theorem}{Theorem}[section] 
\newtheorem{proposition}{Proposition}[section] 
\newtheorem{definition}{Definition}[section] 
\newtheorem{lemma}{Lemma}[section] 
\newtheorem{notation}{Notation}[section] 
\newtheorem{remark}{Remark}[section] 
\newtheorem{corollary}{Corollary}[section] 
\newtheorem{terminology}{Terminology}[section] 
\newtheorem{example}{Example}[section] 
\numberwithin{equation}{section}

\DeclareMathOperator{\diam}{diam}
\DeclareMathOperator{\rk}{rk}
\DeclareMathOperator{\rank}{rank}
\DeclareMathOperator{\Hom}{Hom}
\DeclareMathOperator{\Dom}{Dom}
\DeclareMathOperator{\grad}{grad}
\DeclareMathOperator{\Span}{Span}
\DeclareMathOperator{\interior}{int}
\DeclareMathOperator{\ind}{ind}
\DeclareMathOperator{\supp}{supp}
\DeclareMathOperator{\sgn}{sgn}
\DeclareMathOperator{\ob}{ob}
\DeclareMathOperator{\Spec}{Spec}
\DeclareMathOperator{\PreSh}{PreSh}
\DeclareMathOperator{\Fun}{Fun}
\DeclareMathOperator{\Ker}{Ker}
\DeclareMathOperator{\Image}{Im}
\DeclareMathOperator{\Ad}{Ad}
\DeclareMathOperator{\ad}{ad}
\DeclareMathOperator{\End}{End}
\DeclareMathOperator{\GL}{GL}
\DeclareMathOperator{\SL}{SL}
\DeclareMathOperator{\SO}{SO}
\DeclareMathOperator{\Lie}{Lie}
\DeclareMathOperator{\tr}{tr}


\title{Representation Theory 1 V4A3 Sheet 5 Exercise 1}
\author{So Murata}
\date{2024/2025 Winter Semester - Uni Bonn}

\begin{document}
\section*{Exercise 1}

\subsection*{(i)}
For any $g\in G$ we have
\begin{equation*}
b(\pi(g)x,\pi(g)y) = b(x,y).
\end{equation*}
Thus, if we regard $\Phi(g)\to b(\pi(g)x,\pi(g)y)$ as a function, this is a constant function mapping $g\in G$ to $b(x,y)$. Therefore, its derivative is $0$.\\
\par Let $X\in L$ and calculate the derivative of $\Phi$ with respect to $X$, by the bilinearness of $b$ we get
\begin{equation*}
b(\pi(1+X)x,\pi(1+X)y) - b(x,y) = b(\pi(X)x,y)+b(x,\pi(X)y)+b(\pi(X)x,\pi(X)y)\mod{o(X)}.
\end{equation*}
This is equal to $0$, by taking at least second order terms we derive
\begin{equation*}
b(d\pi(X)x,y)+b(x,d\pi(X)y)=0.
\end{equation*}

\subsection*{(ii)}

Since $V$ is finite dimensional, $\mathfrak{gl}(V)$ is also finite dimensional. Therefore, trace is symmetric bilinear.\\
\par Observe that $GL(\mathfrak{gl}(V)) = GL(V)$ by linearlity of automorphisms as algebras and finite dimension arguments. Let
\begin{equation*}
\pi:GL(V)\to GL(V), \mathfrak{gl}(V)\ni X\mapsto gXg^{-1}.
\end{equation*}
Then its derivative at $1$ is 
\begin{equation*}
d\pi(1)(X) = [X,\cdot].
\end{equation*}
From the tools of linear algebra we have
\begin{equation*}
\tr(gXg{-1}gYg^{-1}) = \tr(gXYg^{-1}) = \tr(gg^{-1}XY) = \tr(XY).
\end{equation*}
By using the previous problem, we conclude $\tr$ is invariant.
\subsection*{(iii)}

Let $\ad_I(X)$ be the adjoint representation on the subalgebra $I$. Since $L$ is finite dimensional, so is $I$. By extending the basis of $I$ to $L$, we get a matrix
\begin{equation*}
\begin{pmatrix}
\ad_I(X)&*\\
O&O
\end{pmatrix}
\end{equation*}
which is the matrix expression of $\ad_I(X)$ in $L$. The lower parts are $O$ because $I$ is an ideal. This equals to the matrix expression of $\ad(X)$ in $L$ by the base change. By multiplying this for $Y$ we get
\begin{equation*}
\begin{pmatrix}
\ad_I(X)\circ\ad_I(Y)&*\\
O&O
\end{pmatrix}
\end{equation*}
which is equal to 
\begin{equation*}
\ad_L(X)\circ\ad_L(Y)
\end{equation*}
Therefore, we conclude the statement.
\subsection*{(iv)}
Since $Z(L)$ is an ideal, we can extend a basis of it to get the basis in $L$. With respect to such basis we get
\begin{equation*}
\ad_L(X)=
\begin{pmatrix}
\ad_J(X)&O\\
O&O
\end{pmatrix}
\end{equation*}
Where $J$ is the subspace of $L$ generated by the basis elements not in the center. Thus by the projection we get
\begin{equation*}
\pi\circ\ad_{L/Z(L)} = \ad_J.
\end{equation*}
\end{document}