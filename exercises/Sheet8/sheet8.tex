\documentclass{article}

\usepackage{amsmath}
\usepackage{amssymb}
\usepackage{amsthm}
\usepackage{enumerate}
\usepackage{bbm}
\usepackage{lipsum}
\usepackage{fancyhdr}
\usepackage{calrsfs}
\usepackage{tikz-cd} 

\newtheorem{theorem}{Theorem}[section] 
\newtheorem{proposition}{Proposition}[section] 
\newtheorem{definition}{Definition}[section] 
\newtheorem{lemma}{Lemma}[section] 
\newtheorem{notation}{Notation}[section] 
\newtheorem{remark}{Remark}[section] 
\newtheorem{corollary}{Corollary}[section] 
\newtheorem{terminology}{Terminology}[section] 
\newtheorem{example}{Example}[section] 
\numberwithin{equation}{section}

\DeclareMathOperator{\diam}{diam}
\DeclareMathOperator{\rk}{rk}
\DeclareMathOperator{\rank}{rank}
\DeclareMathOperator{\Isom}{Isom}
\DeclareMathOperator{\Hom}{Hom}
\DeclareMathOperator{\Dom}{Dom}
\DeclareMathOperator{\grad}{grad}
\DeclareMathOperator{\Span}{Span}
\DeclareMathOperator{\interior}{int}
\DeclareMathOperator{\ind}{ind}
\DeclareMathOperator{\supp}{supp}
\DeclareMathOperator{\sgn}{sgn}
\DeclareMathOperator{\ob}{ob}
\DeclareMathOperator{\Spec}{Spec}
\DeclareMathOperator{\PreSh}{PreSh}
\DeclareMathOperator{\Fun}{Fun}
\DeclareMathOperator{\Ker}{Ker}
\DeclareMathOperator{\Image}{Im}
\DeclareMathOperator{\Ad}{Ad}
\DeclareMathOperator{\ad}{ad}
\DeclareMathOperator{\End}{End}
\DeclareMathOperator{\GL}{GL}
\DeclareMathOperator{\SL}{SL}
\DeclareMathOperator{\SU}{SU}
\DeclareMathOperator{\Lie}{Lie}
\DeclareMathOperator{\tr}{tr}
\DeclareMathOperator{\Der}{Der}
\DeclareMathOperator{\Aut}{Aut}
\DeclareMathOperator{\rad}{rad}
\DeclareMathOperator{\id}{id}

\newcommand{\tens}[1]{%
  \mathbin{\mathop{\otimes}\displaylimits_{#1}}%
}

\title{Representation Theory 1 V4A3}
\author{So Murata}
\date{2024/2025 Winter Semester - Uni Bonn}

\begin{document}
\maketitle

\section*{Exercise 3}

\subsection*{(iii)}

Let $g\in SU(4)$ and $u\wedge v\in\bigwedge^2\mathbb{C}^4$, we let
\begin{equation*}
g(u\wedge v) = gu\wedge gv.
\end{equation*}
For $v\in\bigwedge^2\mathbb{C}^4$ we define $v^*\in\mathbb{C}^4$ by 
\begin{equation*}
u\wedge v^* = \langle u,v\rangle e_1\wedge e_2\wedge e_3\wedge e_4,
\end{equation*}
where $\langle\cdot,\cdot\rangle$ is the Hermitian inner product induced by standard Hermitian inner product on $\mathbb{C}^4$.

For $i,j,k,l\in\{1,2,3,4\}$ and each of them is distinct, we have
\begin{equation*}
b_{ij,kl} = {\frac 1 {\sqrt{2}}}(e_i\wedge e_j+e_k\wedge e_l),\quad \overline{b}_{ij,kl} = {\frac i {\sqrt{2}}}(e_i\wedge e_j- e_k\wedge e_l).
\end{equation*}
Then this forms a basis in $\bigwedge^+\mathbb{C}^4$. Obviously $SU(4)$ preserves the inner product so maps into $O(6)$, and as $SU(4)$ is connected image lies in $SO(6)$. By the first equation, above, we have the homomorphism between $SU(4)$ and $SO(6)$. 
\end{document}