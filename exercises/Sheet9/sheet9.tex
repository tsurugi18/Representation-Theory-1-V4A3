\documentclass{article}

\usepackage{amsmath}
\usepackage{amssymb}
\usepackage{amsthm}
\usepackage{enumerate}
\usepackage{bbm}
\usepackage{lipsum}
\usepackage{fancyhdr}
\usepackage{calrsfs}
\usepackage{tikz-cd} 

\newtheorem{theorem}{Theorem}[section] 
\newtheorem{proposition}{Proposition}[section] 
\newtheorem{definition}{Definition}[section] 
\newtheorem{lemma}{Lemma}[section] 
\newtheorem{notation}{Notation}[section] 
\newtheorem{remark}{Remark}[section] 
\newtheorem{corollary}{Corollary}[section] 
\newtheorem{terminology}{Terminology}[section] 
\newtheorem{example}{Example}[section] 
\numberwithin{equation}{section}

\DeclareMathOperator{\diam}{diam}
\DeclareMathOperator{\rk}{rk}
\DeclareMathOperator{\rank}{rank}
\DeclareMathOperator{\Isom}{Isom}
\DeclareMathOperator{\Hom}{Hom}
\DeclareMathOperator{\Dom}{Dom}
\DeclareMathOperator{\grad}{grad}
\DeclareMathOperator{\Span}{Span}
\DeclareMathOperator{\interior}{int}
\DeclareMathOperator{\ind}{ind}
\DeclareMathOperator{\supp}{supp}
\DeclareMathOperator{\sgn}{sgn}
\DeclareMathOperator{\ob}{ob}
\DeclareMathOperator{\Spec}{Spec}
\DeclareMathOperator{\PreSh}{PreSh}
\DeclareMathOperator{\Fun}{Fun}
\DeclareMathOperator{\Ker}{Ker}
\DeclareMathOperator{\Image}{Im}
\DeclareMathOperator{\Ad}{Ad}
\DeclareMathOperator{\ad}{ad}
\DeclareMathOperator{\End}{End}
\DeclareMathOperator{\GL}{GL}
\DeclareMathOperator{\SL}{SL}
\DeclareMathOperator{\SU}{SU}
\DeclareMathOperator{\Lie}{Lie}
\DeclareMathOperator{\tr}{tr}
\DeclareMathOperator{\Der}{Der}
\DeclareMathOperator{\Aut}{Aut}
\DeclareMathOperator{\rad}{rad}
\DeclareMathOperator{\id}{id}
\DeclareMathOperator{\height}{ht}

\newcommand{\tens}[1]{%
  \mathbin{\mathop{\otimes}\displaylimits_{#1}}%
}

\title{Representation Theory 1 V4A3 Sheet 9}
\author{So Murata}
\date{2024/2025 Winter Semester - Uni Bonn}

\begin{document}
\maketitle
\section{Exercise 3}
The faithfulness of the action is immediate from the fact that $T$ is maximal abelian. 
Since $\mathfrak{g}_{\alpha}$ is of dimension $1$, we will prove that $\Ad(n):\mathfrak{g}_{\alpha}\to\mathfrak{g}_{\beta}$ is an injection. \\\par
Let us note that we have for all $t\in T$, 
\begin{equation*}
X\in \mathfrak{g}_\alpha\Leftrightarrow \forall t\in \mathfrak{t}_{\mathbb{C}}, \ad(t)X = \alpha(t)X.
\end{equation*}
Substituting $\Ad(n)X$ for $X$ we get
\begin{equation*}
\ad(t) X = \alpha(t)\Ad(n)^{-1}\Ad(n)X.
\end{equation*}
This is well-defined as $\Ad(n)$ is an isomorphism and $n$ is a normalizer of $T$.\\
\par Since the action is injective and the root system $R$ is finite, there are at most $|R|!$ many ways of such actions as they represent permutations, we have that $|W|$ is also bounded by $|R|!$. 
\end{document}