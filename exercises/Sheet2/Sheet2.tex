\documentclass{article}

\usepackage{amsmath}
\usepackage{amssymb}
\usepackage{amsthm}
\usepackage{enumerate}
\usepackage{bbm}
\usepackage{lipsum}
\usepackage{fancyhdr}
\usepackage{tikz-cd} 

\newtheorem{theorem}{Theorem}[section] 
\newtheorem{proposition}{Proposition}[section] 
\newtheorem{definition}{Definition}[section] 
\newtheorem{lemma}{Lemma}[section] 
\newtheorem{notation}{Notation}[section] 
\newtheorem{remark}{Remark}[section] 
\newtheorem{corollary}{Corollary}[section] 
\newtheorem{terminology}{Terminology}[section] 
\newtheorem{example}{Example}[section] 
\numberwithin{equation}{section}

\DeclareMathOperator{\diam}{diam}
\DeclareMathOperator{\rk}{rk}
\DeclareMathOperator{\rank}{rank}
\DeclareMathOperator{\Hom}{Hom}
\DeclareMathOperator{\Dom}{Dom}
\DeclareMathOperator{\grad}{grad}
\DeclareMathOperator{\Span}{Span}
\DeclareMathOperator{\interior}{int}
\DeclareMathOperator{\ind}{ind}
\DeclareMathOperator{\supp}{supp}
\DeclareMathOperator{\sgn}{sgn}
\DeclareMathOperator{\ob}{ob}
\DeclareMathOperator{\Spec}{Spec}
\DeclareMathOperator{\PreSh}{PreSh}
\DeclareMathOperator{\Fun}{Fun}
\DeclareMathOperator{\Ker}{Ker}


\title{Representation Theory 1 V4A3 Exercise Sheet 2 Problem 3}
\author{So Murata}
\date{2024/2025 Winter Semester - Uni Bonn}

\begin{document}
\maketitle

\section*{(1)}

Let $\alpha(t)=\Phi_v(t+s,p)$ and $\beta(t) = \Phi_v(t,\Phi_v(s,p))$. Then for a fixed $s$ and $t=0$, we have
\begin{equation*}
\alpha(0) = \Phi_v(s,p) = \Phi_v(0,\Phi_v(s,p))=\beta(0).
\end{equation*}

Also we have for some neighborhood, 
\begin{equation*}
{\frac {d\alpha} {dt}}(t) = v(\alpha(t)), \quad {\frac {d\beta} {dt}}(t) = v(\beta(t)).
\end{equation*}

By the uniqueness of $c_{p,v}$, we derive that $\alpha=\beta$.\\
\par The above also proves for $s$, since $s$ can be picked arbitrary and they are equal at $t=0$.

\section*{(2)}

Let $\beta(t) = R_p\Phi_v(t,1)$ and $\alpha(t) = R_p\beta(t)$ then $\alpha(0)=p$. We need to show that $v(\alpha(t)) = dR_{\alpha(t)}(1)X$. Indeed we have
\begin{equation*}
dR_{pq}(1) = d(R_q\circ R_p)(1) = dR_q(R_p(1))dR_p(1).
\end{equation*}

Using this we derive
\begin{align*}
(\alpha)'(t) &= dR_p(\beta(t))dR_{\beta(t)}(1)X\\
& = dR_p(R_{\beta(t)}1)dR_{\beta(t)}(1)X\\
& = dR_{\beta(t)p}(1)X\\
&= dR_{\alpha(t)}(1)X.
\end{align*}

By the uniqueness, we get $\alpha(t) = \Phi_v(t,p)$.

\end{document}