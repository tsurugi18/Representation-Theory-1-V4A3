\documentclass{article}

\usepackage{amsmath}
\usepackage{amssymb}
\usepackage{amsthm}
\usepackage{enumerate}
\usepackage{bbm}
\usepackage{lipsum}
\usepackage{fancyhdr}
\usepackage{calrsfs}
\usepackage{tikz-cd} 

\newtheorem{theorem}{Theorem}[section] 
\newtheorem{proposition}{Proposition}[section] 
\newtheorem{definition}{Definition}[section] 
\newtheorem{lemma}{Lemma}[section] 
\newtheorem{notation}{Notation}[section] 
\newtheorem{remark}{Remark}[section] 
\newtheorem{corollary}{Corollary}[section] 
\newtheorem{terminology}{Terminology}[section] 
\newtheorem{example}{Example}[section] 
\numberwithin{equation}{section}

\DeclareMathOperator{\diam}{diam}
\DeclareMathOperator{\rk}{rk}
\DeclareMathOperator{\rank}{rank}
\DeclareMathOperator{\Hom}{Hom}
\DeclareMathOperator{\Dom}{Dom}
\DeclareMathOperator{\grad}{grad}
\DeclareMathOperator{\Span}{Span}
\DeclareMathOperator{\interior}{int}
\DeclareMathOperator{\ind}{ind}
\DeclareMathOperator{\supp}{supp}
\DeclareMathOperator{\sgn}{sgn}
\DeclareMathOperator{\ob}{ob}
\DeclareMathOperator{\Spec}{Spec}
\DeclareMathOperator{\PreSh}{PreSh}
\DeclareMathOperator{\Fun}{Fun}
\DeclareMathOperator{\Ker}{Ker}
\DeclareMathOperator{\Image}{Im}
\DeclareMathOperator{\Ad}{Ad}
\DeclareMathOperator{\ad}{ad}
\DeclareMathOperator{\End}{End}
\DeclareMathOperator{\GL}{GL}
\DeclareMathOperator{\SL}{SL}
\DeclareMathOperator{\SO}{SO}
\DeclareMathOperator{\Lie}{Lie}


\title{Representation Theory 1 V4A3 Sheet 4 Exercise 1}
\author{So Murata}
\date{2024/2025 Winter Semester - Uni Bonn}

\begin{document}
\maketitle
\section*{Exercise 2}

\subsection*{(i)}

Since the characteristic polynomial of $M$ is of degree three in real coefficient, it has at least one real solution $\lambda$. Therefore $M$ is similar to the matrix
\begin{equation*}
\begin{pmatrix}
R&\underline{o}\\
\underline{o}^T&\lambda
\end{pmatrix}.
\end{equation*}
Since $M$ is orthogonal, we have $M^TM=I$, this means that $\det R=1$ and therefore $\lambda = 1$. By this construction, it obvious that if there are two linearly independent vectors which are the eigenvectors of $M$ with eigenvalue $1$, $M$ is the identity matrix.

\subsection*{(ii)}

Let us define a map
\begin{equation*}
\Phi:B\to S^3, \Phi(\psi,\theta,\phi) = (\cos\psi,\sin\psi\cos\theta,\sin\psi\sin\theta\cos\phi,\sin\psi\sin\theta\sin\phi)
\end{equation*}
is the global diffeomorphism between $B$ and $S^3$. \\
\par Let $\pi:B\to B/\sim$ to be a canonical mapping, then this is a quotient map since open set in $U$ of $B$, we have
\begin{equation*}
\pi(U) = U\sqcup (-U),
\end{equation*}
where $-U$ is the sets of points $(-x,-y,-z)$ in polar coordinate where there exists $(\psi,\theta,\phi)$ represents $(x,y,z)$ in $U$.\\
\par Let us now define a map 
\begin{equation*}
(\psi,\theta,\phi)\mapsto
T_{E\to C}
\begin{pmatrix}
1&\underline{o}^T\\
\underline{o}&R_\psi
\end{pmatrix}
T_{C\to E}
\end{equation*}
where $E$ is the standard basis of $\mathbb{R}^3$ and $C$ is the matrix obtained by Gram-Schmidt orthonormalization started from the normalized vector of $(\psi,\theta,\phi)$ in the cartesian coordinate. \\
\par Then this represents a rotation around a line passing through the origin and $(\psi,\theta,\phi)$. This is a bijection between $B/\sim$ and $\SO(3)$ by the theorem of isometries in three dimensional space.\\
\par Let us define an operation by 
\begin{equation*}
(\psi_1,\theta_1,\phi_1)\cdot(\psi_2,\theta_2,\phi_2) = (\psi_1+\psi_2\mod \pi,\theta,\phi)
\end{equation*}
where the line passing through the origin and $(\psi_1+\psi_2\mod \pi,\theta,\phi)$ is orthogonal to both lines passing through $(\psi_1,\theta_1,\phi_1),(\psi_2,\theta_2,\phi_2)$ and the origin if the two lines form a plane. Otherwise, define a rotation with the angle the sum of two rotations. This defines a group operation on $B/\sim$. Furthermore, this makes the bijection into an isomorphism since each rotation is represented as a product of two reflections. 

\subsection*{(iii)}

The map
\begin{equation*}
\pi:B\to B/\sim
\end{equation*}
is a covering since for an arbitrary point $\underline{x}\in B$. For any ball $U$ around $x$ which does not contain the origin, take $\pi(U)$. Then we have
\begin{equation*}
\pi^{-1}(\pi(U)) = U\sqcup -U.
\end{equation*}
And any ball $U$ around the origin, we have
\begin{equation*}
\pi^{-1}(\pi(U)) = U.
\end{equation*}
Therefore, $\pi$ is a covering. Therefore, we can make $B$ into a connected Lie group. We know this is diffeomorphic to $S^3$ by $\Phi$. We claim the statement.
\end{document}