\documentclass{article}

\usepackage{amsmath}
\usepackage{amssymb}
\usepackage{amsthm}
\usepackage{enumerate}
\usepackage{bbm}
\usepackage{lipsum}
\usepackage{fancyhdr}
\usepackage{calrsfs}
\usepackage{tikz-cd} 

\newtheorem{theorem}{Theorem}[section] 
\newtheorem{proposition}{Proposition}[section] 
\newtheorem{definition}{Definition}[section] 
\newtheorem{lemma}{Lemma}[section] 
\newtheorem{notation}{Notation}[section] 
\newtheorem{remark}{Remark}[section] 
\newtheorem{corollary}{Corollary}[section] 
\newtheorem{terminology}{Terminology}[section] 
\newtheorem{example}{Example}[section] 
\numberwithin{equation}{section}

\DeclareMathOperator{\diam}{diam}
\DeclareMathOperator{\rk}{rk}
\DeclareMathOperator{\rank}{rank}
\DeclareMathOperator{\Hom}{Hom}
\DeclareMathOperator{\Dom}{Dom}
\DeclareMathOperator{\grad}{grad}
\DeclareMathOperator{\Span}{Span}
\DeclareMathOperator{\interior}{int}
\DeclareMathOperator{\ind}{ind}
\DeclareMathOperator{\supp}{supp}
\DeclareMathOperator{\sgn}{sgn}
\DeclareMathOperator{\ob}{ob}
\DeclareMathOperator{\Spec}{Spec}
\DeclareMathOperator{\PreSh}{PreSh}
\DeclareMathOperator{\Fun}{Fun}
\DeclareMathOperator{\Ker}{Ker}
\DeclareMathOperator{\Image}{Im}
\DeclareMathOperator{\Ad}{Ad}
\DeclareMathOperator{\ad}{ad}
\DeclareMathOperator{\End}{End}
\DeclareMathOperator{\GL}{GL}
\DeclareMathOperator{\SL}{SL}
\DeclareMathOperator{\SU}{SU}
\DeclareMathOperator{\Lie}{Lie}
\DeclareMathOperator{\tr}{tr}
\DeclareMathOperator{\Der}{Der}
\DeclareMathOperator{\Aut}{Aut}
\DeclareMathOperator{\rad}{rad}
\DeclareMathOperator{\Sym}{Sym}

\newcommand{\tens}[1]{%
  \mathbin{\mathop{\otimes}\displaylimits_{#1}}%
}

\title{Representation Theory 1 V4A3 Sheett 6}
\author{So Murata}
\date{2024/2025 Winter Semester - Uni Bonn}

\begin{document}
\maketitle

\section*{Exercise 2}

We have the homeormophism

\begin{equation*}
\mathfrak{su}_2(\mathbb{R}\tens{\mathbb{R}}\mathbb{C} \cong \mathfrak{sl}_2(\mathbb{C}).
\end{equation*}

And the fact that for any vector space $V$ and its subspace $W\subseteq V$,
\begin{equation*}
W\text{ is $\mathfrak{sl}_2(\mathbb{C})$ invariant.}\Leftrightarrow W\text{ is $\mathfrak{su}_2(\mathbb{R})$ invariant.}
\end{equation*}

Since $\mathfrak{su}_2(\mathbb{R})$ is connected and simply connected, for any representation $(\rho,V)$ there is a representation $(\pi,V)$ of $\SU_2(\mathbb{R})$ such that 
\begin{equation*}
d\pi(1) =\rho.
\end{equation*}

For any $U\in\SU_2(\mathbb{R})$, there is $X\in\mathfrak{su}_2(\mathbb{R})$ and 
\begin{equation*}
\pi(U) = \exp(\rho(X)).
\end{equation*}

From the elementary linear algebra, we have $U$ is diagonalizable. This is preserved under a group homomorphism thus $\pi(U)$
is diagonalizable. Again from the elementary linear algebra, we have
\begin{equation*}
\exp(\rho(X))\text{ is diagonalizable.}\Leftrightarrow \rho(x)\text{ is diagonalizable.}
\end{equation*}

By the previous argumentt $\rho(h)$ is diagonalizable. Let $V_1$ be an eigenspace of eigenvalues of $\lambda_1$ of $e$.

Since $\rho$ preserves the bracket we have
\begin{align*}
[\rho(h),\rho(e)]V_1 = 2\rho(e)V_1\Rightarrow \rho(h)\rho(e)V_1 = (2+\lambda_1)\rho(e)V_1, \\
[\rho(h),\rho(f)]V_1 = -2\rho(f)V_1\Rightarrow \rho(h)\rho(f)V_1 = (-2+\lambda_1)\rho(f)V_1. 
\end{align*}

Solving the system of linear equation, we conclude that it is impossible unless $\rho(e)V_1,\rho(f)V_1$ are $\{o\}$.\\

\par Let us consider an action of $\SU_2(\mathbb{C})$ on $\mathbb{C}[x,y]$, which is by the base change such that for any $f\in\mathbb{C}[v_1,v_2]$ and $P\in\SU_2(\mathbb{C})$ we have
\begin{equation*}
P=\begin{pmatrix}a&b\\c&d\\\end{pmatrix},\quad f(v_1,v_2)P = f(av_1+cv_2,bv_1+dv_2).
\end{equation*}
\par Denote $V_n$ as the set of homogeneous polynomials of degree $n$. Then we claim this is an invariant subspace of the action. This is because such action is homogeneous linear transformation.\\

\par We learned that the basis of $V=\Sym^n(\mathbb{C}^2)$ is
\begin{equation*}
\{v_1^iv_2^{n-i}\}_{i=0,\cdots,n}.
\end{equation*}

However, for each $v_1^iv_2^{n-i}$ we have
\begin{equation*}
hv_1^iv_2^{n-i} = (n-2i)v_1^iv_2^{n-i}.
\end{equation*}

\par Thus the eigenspace of $\rho(h)$ is the whole space, and thus $\rho(e)V,\rho(f)V=\{o\}$. Thus $V$ is a irreducible representation of $\mathfrak{sl}_2(\mathbb{C})$.\\

\par On the other hands, if $V$ is an irreducible representation of $\mathfrak{sl}_2(\mathbb{C})$, then $\rho(h)$ is a multiple of an identity and $\rho(e),\rho(f)=O$. Thus we conclude the statement holds.

\end{document}